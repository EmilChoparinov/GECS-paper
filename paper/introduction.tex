\section{Introduction}

The following is the introduction to the papers goals and motivations.

\subsubsection{Research Question}

The core research question of this paper is as follows:

\blockquote{Can you design a concurrently Entity Component System such that it is wait-free, and what are the advantages and trade-offs of such a system?}

This paper explores various different ECS implementation styles, going through their advantages, disadvantages, and concurrency capabilities before settling on using a Dense ECS. Using this style, the paper then explores how to organize objects in the simulation to maximize concurrency without using locks. 

Those ideas for object organization are then formulated into a concurrency model in the theory section before moving onto its implementation. The concurrency model helps define specific properties and exhibits behaviors that are helpful guides when deep into implementing a solution. 

Using this theory, the paper implements the model into the GECS C Library, short for "Graph Entity Component System", which extends the theory into a graph of state machines to manage and maintain objects.

\subsubsection{Motivation}

I would not be surprised if there already exists a wait-free ECS that would satisfy the research question but throughout the discovery phase, I was unable to find one that advertised itself as such. Many ECS's that were studied mentioned their concurrency only in a passing sense and without good documentation from these developers, it's hard to tell if they offered a wait-free solution.

Regardless, this paper is of my personal design and ideas, built around the concept of archetypes inspired by Sander Merterns and his FLECS ECS.

Personally, I wanted to set out to build an ECS that hopefully will lead to more research or improve existing or in-development ECS's concurrency models. This paper discusses the 5th iteration of the GECS ECS, and the theory built for it to help me grow the project in the right direction.

\newpage