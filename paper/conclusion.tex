\section{Conclusion}
To conclude, the question of "Can you make a wait-free ECS?" was answered as this paper proposes a model that is wait-free. What started off as a practical project, which was intended to be the main meat of the paper, led to the development of some interesting theory. The difficulties in benchmarking the model show that this ECS is a little unique in its concurrency architecture and that it is missing key production features. In my opinion, I do not believe the benchmark results are accurate measures of this engine's strengths because the benchmarks focused on processing as many entities as quickly as possible in specialized settings. This engine, with its finite state machine, has the ability to grow, which is unique. A more accurate test for GECS would be to implement the same set of fairly complex game systems in different engines and test via throughput on how fast a tick processes.

In summary, while the proposed wait-free ECS model shows promise and introduces an alternative approach, further work is necessary to refine its entity throughput and to establish more comprehensive benchmarking methods. More ongoing research could lead to improvements to the efficiency of simulations, potentially influencing future developments.
\newpage