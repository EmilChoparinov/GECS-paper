\section{GECS - Graph Entity Component System}
\label{chap:3}

The following chapter introduces the GECS C Library, built with the theory introduced in chapter \ref{chap:2}. Not all code will be presented in this paper, but the implementation is freely available online \href{https://github.com/EmilChoparinov/GECS}{here}. All code presented is written by me except for the hashing function and contains zero dependencies and is C99 compatible.

\subsection{Inspiration}
Before diving into the implementation, it's important to know how the theory connects with this library. GECS stands for Graph Entity Component System so obviously the inspiration is graphs. 

The graph part comes from the fact that the overall structure of the ECS is recursive. Instead of holding a ledger, each archetype simulates where entities are supposed to transition to instead. This means that on the Entity FSM, each archetype contains their own Entity FSM. The "real" world context holds the true Entity FSM while each other is a "simulated" world context. Roughly, at each tick the ledgers on each archetype gets their non-concurrent data dumped, transitioned, and ready for the next tick.

As such, GECS creates visual divisions in structures on what can be parallelized and what can't. The graph GECS is modeled from is shown in Figure \ref{fig:gecs_fsm}.

\begin{figure}[H]
    \centering
    \begin{verbatim}                                                                            
                       +------+    +------+     +---- Empty                                 
                       | +--+ |    | +--+ |     |     FSM                           
                       | |{}| |    | |{}| |<----+                                     
                    +->| +--+ | +->| +--+ |                                      
                    |  +------+ |  +------+                                      
                 +--|-----------+-----------+                                    
                 |  |           |           |                                    
                 |  |  +---+    |           |  +--------------+                  
                 |  |  |{A}|----+           |  |              |                  
                 |  |  +---+                |  | +---+        |                  
                 |  |    ^                  |  | |{A}|        |                  
                 |  |    |                  |  | +---+        |                  
                 |  |    | +--+     +-----+ |  |  ^           |                  
                 |  |    +-|{}|---->|{A,C}|--->|  |           |                  
                 |  |      +--+     +-----+ |  | +--+   +---+ |                  
                 | ++--+    |               |  | |{}|-->|{B}| |                  
                 | |{B}| <--+               |  | +--+   +---+ |                  
                 | +---+             "Real" |  |  "Simulated" |                  
                 |                      FSM |  |          FSM |                  
                 +--------------------------+  +--------------+                
   \end{verbatim}
    \caption{Example GECS FSM}
    \label{fig:gecs_fsm}
\end{figure}

In Figure \ref{fig:gecs_fsm}, the graph displays the following properties:
\begin{enumerate}
    \item There is always exactly one "real" context.
    \item All archetype vertices can contain edges to other archetype vertices.
    \item All simulated archetype vertices always contains an edge to the real context.
    \item All real archetype vertices always contain an edge to one simulated context.
    \item Each simulated context can only connect to one archetype from the real context.
    \item A simulated archetype cannot have an edge to another simulated archetype.
\end{enumerate}

The key feature of the GECS library is that each archetype simulates the events it wants to apply to the sequential world before the serialization point occurs. After the serialization point, the sequential world takes over and performs an analysis on the changes within each archtype to update itself before starting the next frame.

Effectively, the simulated worlds are the ledgers discussed in section \ref{sec:ledger} with the added benefit instead of caching operations to be done in parallel, the world simulates as if those changes are actually happening. So at the serialization point, the only operation that needs to be performed is a $\delta$ transition to another archetype for entities in the ledger, alognside new data.

\subsection{Data Structures}
\label{sec:ds}
The datastructures that compose the GECS library are thread-unsafe vectors and hashmaps. These datastructures are provided by another library self-developed called "csdsa", which contains C implementations of DSA fundamentals in C. 

The vector in csdsa wraps contiguous elements and provides a set of functional programming style utility functions to help with code clarity. The hashmap is implemented as an open address hashing table with a load factor of $0.75$. The set implementation is the same as the hashmap. There is nothing special about these structures. Originally, the hashmap was thread safe and was implemented using this paper \cite{hashmaps} but later found it unnecessary so it was scraped in development. 

\subsection{Allocator Design}
The allocator used internally by the GECS C Library is a customized stack allocator. This allocator was chosen because of how easy it is to implement concurrently. The stack allocator is not fancy in anyway, but supports creating stack frames in the heap to keep allocation requests minimized. GECS starts by allocating 8KB of memory for internal stack use. This memory is not used for allocating \texttt{g\_core\_t} members or other long-persisting objects. This memory is only used for internal manipulations and overall just available for routines when needed.

\subsection{World Struct}
The code defined in Figure \ref{code:g_core_t} is not exactly the same code as seen on the repository on GitHub. In the actual code, the types are different in order for the C compiler to do it's magic. The types presented here are for code clarity instead of compilability.

\texttt{g\_core\_t} represents a world from the concurrency model and it's what's given to the library user to manipulate the ECS.

\subsubsection{Flags}
The \texttt{is\_main} flag checks if this world is the original or if it's a ledger. This is used through various points in the implementation as the recursive breakpoint when doing manipulations to \texttt{g\_core\_t} objects. 

The \texttt{invalidate} flag will make more sense once the archetype struct is introduced but this exists because the \texttt{map\_t} type can grow. Each archetype maintains a cache of addresses for quick access to entity data. When the map resizes, those addresses become invalid and must be recache'd. Even though this is not ideal, the map grows by a factor of 2 each time the load factor is reached, so cache invalidations do not happen often. 

The \texttt{invalidate\_fsm} flag is a flag that gets set whenever a new archetype is added to the graph. This is because the systems must get redistributed across the graph.

\subsubsection{Data}
\texttt{component\_registery} and \texttt{archetype\_registry} are maps that utilize the hashing function. This is because component ids are generated via hashing and archetypes are sets of component ids. In order to ensure the same hash is generated for sets of components in differering orders, the hashes are first sorted before the archetype is passed into a hashing function. Unfortunately there was not enough time to implement a hashset and so ordered vectors are used in their place.

\begin{figure}[htbp]
    \begin{lstlisting}[
        language=C
    ]
struct g_core_t {
  int64_t tick;            /* The current tick being processed */
  atomic_uint_least64_t id_gen; /* Generates unique IDs */
    
  /* FLAGS */
  int8_t is_main;           /* Is "real" or "simulated" */
  int8_t invalidate_fsm;    /* Reprocess FSM before next tick */

  /* DATA */
  map_t component_registry; /* Map: hash(name) -> comp size */
  map_t archetype_registry; /* Map: hash([name]) -> archetype */
  map_t entity_registry;    /* Map: entt id -> archetype id AKA hash([name]) */
    
  vec_t system_registry;    /* Vec: system_data. */
};
    \end{lstlisting}
    \caption{\texttt{g\_core\_t} struct definition}
    \label{code:g_core_t}
\end{figure}

\subsection{Archetype Struct}
The code in Figure \ref{code:archetype} represents the archetype as seen in GECS. There's many members of this so the discussion of them will be split into two groups: The implementation members and the ledger/caching members. 

\begin{figure}[H]
    \begin{lstlisting}[
        language=C
    ]
struct archetype {
  gid archetype_id; /* Unique Identifier for this archetype. */
  
  set_t type_set;      /* Set : [hash(name)] */
  vec_t composite;     /* Contiguous vector of interleaved compents */
  vec_t contenders;    /* Vec: system_data */
  
  /* Entity lookup */
  map_t entt_positions; /* entt id -> index in composite */
  map_t offsets; /* Map: hash(name) -> interleaved comp offset */
  
  /* For concurrency and caching purposes. */
  g_core_t *simulation;          /* OOB mutations go here. */
  
  vec_t     entt_creation_buffer;  /* Queue : entt id */
  vec_t     entt_deletion_buffer;  /* Queue : entt id */
  vec_t     entt_mutation_buffer;  /* Queue : entt */
  vec_t     dead_fragments;          /* Vec : index_of(composite) */
};
    \end{lstlisting}
    \caption{\texttt{archetype} struct definition}
    \label{code:archetype}
\end{figure}

\subsubsection{Archetype Implementation Members}
The \texttt{type\_set} member is a set containing the hashes of type ID's for the components this archetype belongs to. Hashing the bytes of this member produces the \texttt{archetype\_id}. The \texttt{composite} vector is the vector where contiguous elements of this archetype are stored. Each element is a segment of memory in which components are interleaved. The \texttt{offsets} table is a table containing the offset of a particular component being queried on.  

The only member in this set related to concurrency is the \texttt{contenders} vector. This vector contains the list of systems that are processed by this archetype. Each system is processed sequentially. 

\subsubsection{Component Retrieval}
Since all archetypes live inside the composite, each archetype is provided a look-up table where the given component hash name is used to load how many bytes off from the start of the element the component exists on. In Figure \ref{code:component_retrieval}, Component A exists 2 bytes off from the composite element while Component C starts 5 bytes off from the composite element.

\begin{figure}[H]
\begin{verbatim}
Offset Table:                                                             
+-----------+-------+                                                     
|Component  |Offset |                                                     
+-----------+-------+                                                     
|Component A|     2 +------+                                              
+-----------+-------+      |                                              
|Component B|     0 +---+  |                                              
+-----------+-------+   |  |                                              
|Component C|     5 +---+--+--+                                           
+-----------+-------+   |  |  |                                           
                        |  |  |                                           
                        |  |  |                                           
Composite Vector:       v  v  v                                           
+--+--+-----+--+--+-----+--+--+-----+--+--+-----+--+--+-----+--+--+-----+ 
|  |  |     |  |  |     |  |  |     |  |  |     |  |  |     |  |  |     | 
|  |  |     |  |  |     |  |  |     |  |  |     |  |  |     |  |  |     | 
|B.|A.|C....|B.|A.|C....|B.|A.|C....|B.|A.|C....|B.|A.|C....|B.|A.|C....| 
+--+--+-----+--+--+-----+--+--+-----+--+--+-----+--+--+-----+--+--+-----+ 
|Index 0    |Index 1    |Index 2    |Index 3    |Index 4    |Index 5    | 
\end{verbatim}
\caption{Specific Component Retrieval}
\label{code:component_retrieval}
\end{figure}

\subsubsection{Archetype Caching And Ledger Members}
The rest of the members of the \texttt{archetype} struct are members related to caching and concurrency. For doing concurrent entity deletions and creations, the \texttt{entt\_deletion\_buffer} and \texttt{entt\_creation\_buffer} are used respectively. 

Entity transitions like adding and deleting components use the \texttt{entt\_mutation\_buffer}. When an entity gains a new component, the entity is marked to have gained said component by having its entity ID pushed into the mutation buffer. The component is then simulated in the \texttt{simulation}. When it is time to consolidate, the buffer is emptied one by one and their states are transitioned out of the simulation. 

The \texttt{dead\_fragments} vector is also used at consolidation time. This vector contains indices inside the composite vector of entities that no longer use this archetype. They were either deleted or transitioned out. The vector is sorted before use in the defragmentation algorithm.

\begin{figure}[H]
\begin{verbatim}
        +---------+     +----------------------------+                             
        |Real     |     | Archetype:                 |                             
        |Context  |     |                +---------+ |                             
        |         |     |                |Simulated| |                             
        |         |     | Entity         |Context  | |                             
        |         |     | Mutation       |         | |                             
        |     ----+--+  | Buffer:        |         | |                             
        |         |  |  | +----------+   |         | |                             
        |         |  |  | |Entity ID |   |         | |                             
        |         |  |  | +----------+   |         | |                             
        +---------+  |  | |5         |   |         | |                             
                     |  | |          | +-+--->     | |                             
                     |  | |1         | | |         | |                             
                     +--+-+--->  ----+-+ |         | |                             
                        | |13        |   |         | |                             
                        | +----------+   +---------+ |                             
                        |                            |                             
                        +----------------------------+                                                      
\end{verbatim}
\caption{Entity Mutation Interface}
\label{code:component_retrieval}
\end{figure}

\subsection{Systems}
The following is the \texttt{system\_data} struct. The \texttt{system\_data} struct contains a function pointer \texttt{run\_system} which is provided by the user. This is the entry point into the users code. The \texttt{requirements} set is a set of hashes component names where an archetype be a superset to be paired. See section \ref{sec:scheduling}.

\begin{figure}[H]
    \begin{lstlisting}[
        language=C
    ]
struct system_data {
  g_system  run_system;     /* A function pointer to the system. */
  set_t     requirements;   /* Vec: hash(name) */
};
    \end{lstlisting}
    \caption{\texttt{system\_data} struct definition}
    \label{code:sd_and_er}
\end{figure}

\subsection{API Objects}
There are three API structs that library users interface with: The \texttt{g\_pool}, \texttt{g\_par}, and the \texttt{g\_query} structs. 

\texttt{g\_query} is passed as the single parameter to all system functions. Using this struct, a user contextually interfaces with the system currently in process. It contains two pointers: a pointer to the world context and a pointer to the archetype.

\texttt{g\_pool} is used by a user who wants to begin querying over components inside the current archetype. It's a context object on how far iteration has progressed. Internally, it's actually just a wrapper over \texttt{g\_pool} and adds an index counter.

\texttt{g\_par} is used by a user who confidently want to do embarassingly parallel tasks using the libaries concurrency functions. It contains references to the offset table, the composite, the archetype, and the current tick.


\begin{figure}[htbp]
    \begin{lstlisting}[
        language=C
    ]
struct g_par {
  vec_t        *stored_components; /* Vector : fragment */
  map_t        *component_offsets; /* Map : hash(comp id) -> pos */
  archetype    *arch;
  int64_t       tick;
};

struct g_pool {
  gint64 idx;
  g_par  entities;
};

struct g_query {
  g_core    *world_ctx;
  archetype *archetype_ctx;
};
    \end{lstlisting}
    \caption{API Object definitions}
    \label{code:apis}
\end{figure}

\subsection{API}
The GECS API is roughly inspired by the FLECS API in how it accesses specific components. Otherwise it is of this papers own design. There are four categories of functions in the GECS API:
\begin{itemize}
    \item World Functions
    \item Thread Unsafe ECS Functions
    \item Thread Safe ECS Functions
    \item Query Functions
\end{itemize}

The header file containing the ECS defintions is linked in the Appendix \ref{appendix:a}.

\subsection{World Functions}
There are three world functions: \texttt{g\_create\_world}, \texttt{g\_progress}, \texttt{g\_destroy\_world} and each are self explanatory. \texttt{g\_progress} runs the ECS engine for exactly 1 tick.

\subsection{Thread Unsafe Functions}
The thread unsafe functions include any function that directly modifies the sequential worlds data structures. Either the maps, vectors, or \texttt{id\_gen}. There exists a thread unsafe function for all the normal entity operations: add, get, set, and remove components. This also includes the create and delete entity operations as well.  

\subsection{Thread Safe Functions}
Just like how there exists a thread unsafe function for all normal entity operations, there exists a thread safe function for all normal entity operations as well. It's also important to note that all query functions are also thread safe.

\subsubsection{Query Functions}
The query functions are the set of functions used to interface and progress the runtime. The two core functions are: \texttt{gq\_vectorize} and \texttt{gq\_seq}. Depending on how the user wants their components processed, they can use the vectorize function to enable concurrent element processing on the contiguous vector or they can use the sequential function to retrieve an iterator. There is nothing stopping the user at this point on creating their own concurrent model inside the system. \texttt{gq\_vectorize} is there only for utility.

\textbf{\texttt{gq\_vectorize} approach:} When using the vectorize function, the only function that can use this structure is the \texttt{gq\_each} function. This function splits the composite to be processed over multiple threads for the user automatically.

\textbf{\texttt{gq\_seq} approach:} This function returns \texttt{g\_pool} structure, allowing the user to process components however they please. To progress to the next element the user invokes \texttt{gq\_next} and to check if there are any elements left the user invokes \texttt{gq\_done}.if there are any elements left call \texttt{gq\_done}.

To select a component, pass in the \texttt{g\_pool} struct with the component name to \texttt{gq\_field}. Alternatively, pass into \texttt{gq\_is\_id} to check against important IDs and to load the hidden archetype components for this entity. 

\subsection{Examples}
In Appendix \ref{appendix:code_example_1}, there exists a demo game showing the usage of GECS. It's a re-creation of that simple chromium browser game seen when there's no internet connection. The majority of the systems run in parallel. The following is a "screenshot" of the game in action:

\begin{figure}[H]
    \begin{verbatim}
===============
│             │
│             │
│ Y         0 │
===============
Health: 19
    \end{verbatim}
\caption{Entity Cache Interface}
\label{code:component_retrieval}
\end{figure}

The game uses the \texttt{ncurses} library for rendering to the terminal. 

\subsection{Entity ID Representation}
In GECS, only entities use ID generation only one ID generator is used throughout the runtime. The ID is syncrhonized using the C atomics library. The reasoning behind \texttt{id\_gen} being least 64 and not fast 64 is because of the 64 byte guarantee was originally important to the library as it did smart ID generation in earlier versions. The last bit used to be reserved to know which context an entity existed in, either real or simulated. But this caused an ID vanishing issue because the user would not get the correct ID returned to them when creating entities concurrently. This essentially meant that any entities made concurrently will not be accessable to the library user.

To mitigate this issue, the ID generator was made atomic. It is kept as least for future development. In the future, GECS should have related entity pools placed in the first 16 bits, much like how FLECS does it.

\texttt{gid.c} contains the implemenation of how GECS increments atomics.

\subsection{The Tick Lifecycle}
The tick lifecycle is processing exactly one tick, which means executing the \texttt{g\_progress} function completely. The cycle is split into three parts: preparation, execution, and cleanup. Both the preparation and cleanup phases are done sequentially on the main thread, while the execution phase is where threads are scheduled to run systems.

The preparation phase only has one job which is to ensure the validity of the entity FSM before systems start transitioning entities over it. This is done by checking the \texttt{reprocess\_fsm} flag.

The execution phase starts by collecting all archetypes and iterating over them to reach their \texttt{contenders} vector. This vector contains pointers to the \texttt{system\_data} structs. Once loaded, threads will be scheduled to run the system process.

The cleanup phase is where the majority of the work occurs for GECS. It must do 4 things:
\begin{itemize}
    \item Check if fsm is invalidated and set invalidate flag
    \item Migrate entities out of their simulations
    \item Process the buffers and clear them to prepare for the next tick
    \item Defragment all composites that have unreference-able indices.
\end{itemize}

The cleanup phase algorithms for defragmentation and cache clearance are written into this section. The following sections will contain the algorithms for entity migration, Entity FSM processing, and scheduling that were referenced here.

\subsubsection{Defragmentation}
Defragmentation is the process of re-vectorizing a composite such that all elements in the vector are referenced by one lookup. In order to achieve this: the entities who have either been deleted or transitioned away from this archetype must have their composite index loaded and then removed from the vector. There are two parts to the defragmentation algorithm.

Because classical vector deletion is $O(N)$ time, if $e$ amount of entities are deleted, then the defragmentation algorithm will take $O(N^e)$ which is not feasible. As such, the vector \texttt{dead\_fragments} is used. When an entity leaves this archetype, the dead index is pushed into \texttt{dead\_fragments}. Then, at cleanup, the following algorithm is used to defragment the vector in $O(log(e) + N)$ time for $e$ indices.

\begin{figure}[H]
    \begin{enumerate}
        \item Sort \texttt{dead\_index}
        \item Create vector \texttt{out}
        \item Create vector \texttt{rolling\_offsets} and integer \texttt{dead\_index}.
        \item For each element $ e \in \texttt{composite}$, do the following 5-9:
        \item If the index of $e$ is dead, do the following steps 6-8:
        \item Increment \texttt{dead\_index}
        \item Push $\texttt{rolling\_offsets} \leftarrow \texttt{dead\_index}$
        \item Continue
        \item Push $\texttt{rolling\_offsets} \leftarrow \texttt{dead\_index}$
        \item Replace \texttt{composite} with \texttt{out}
    \end{enumerate}
    \caption{Defragment Algorithm Part 1}
    \label{alg:defrag1}
\end{figure}

In other words, Figure \ref{alg:defrag1}, produces a vector called the \texttt{rolling\_offset} vector that counts up by one each time a dead fragment index is reached. This is then used by part two of the algorithm to subtract all entity positions on this archetype by its associated value at the same index in \texttt{rolling\_offset}.
                  
\begin{figure}[H]
\begin{verbatim}
    Original                                        Dead                     
    Vector                                          Fragments                
    +-----+-----+-----+-----+-----+-----+-----+     +-----+                  
    |  A  |  B  |  C  |  D  |  E  |  F  |  G  | +---+  2  |                  
    +-----+-----+-----+-----+-----+-----+-----+ |   +-----+                  
    0     1     2     3     4     5     6       | +-+  3  |                  
                +-------------------------------+ | +-----+                  
                |                                 | |  5  |                  
    Offset      |     +---------------------------+ +--+--+                  
    Vector      v     v                                |                     
    +-----+-----+-----+-----+-----+-----+-----+        |                     
    |  0  |  0  |  2  |  2  |  2  |  3  |  3  |        |                     
    +-----+-----+-----+-----+-----+-----+-----+        |                     
    0     1     2     3     4     5     6              |                     
                                  ^                    |                     
                                  +--------------------+   
\end{verbatim}
\caption{Offset Vector Diagram}
\label{code:component_retrieval}
\end{figure}

The second part of the algorithm loads the positions and then performs the following: 

\begin{figure}[H]
\begin{verbatim}                                                                 
 Offset Vector                                                                 
   0+-----+     Entity                             New Entity             
    |  0  |     Positions                          Positions                        
   1+-----+     +-----+-----+                      +-----+-----+            
    |  0  |     |ID   |Index|                      |ID   |Index|            
   2+-----+     +-----+-----+ index - offset(index)+-----+-----+            
    |  2  |     |0    |1    +--------------------->|0    |  1  |            
   3+-----+     +-----+-----+ index - offset(index)+-----+-----+            
    |  2  |     |1    |4    +--------------------->|1    |  2  |            
   4+-----+     +-----+-----+ index - offset(index)+-----+-----+            
    |  2  |     |2    |6    +--------------------->|2    |  3  |            
   5+-----+     +-----+-----+                      +-----+-----+            
    |  3  |     |. . .|. . .|                      |. . .|. . .|            
    +-----+     +-----+-----+                      +-----+-----+            
 Original             |                     Defragmented |                  
 Vector+--------------+--------------+      Vector +-----+-----+            
       v                 v           v             v     v     v            
 +-----+-----+-----+-----+-----+-----+-----+ +-----+-----+-----+-----+      
 |  A  |  B  |  C  |  D  |  E  |  F  |  G  | |  A  |  B  |  E  |  G  |      
 +-----+-----+-----+-----+-----+-----+-----+ +-----+-----+-----+-----+      
 0     1     2     3     4     5     6       0     1     2     3            
\end{verbatim}
\caption{Update Entity Record With Offset Mapping Diagram}
\label{code:component_retrieval}
\end{figure}

This was part two of the defragmentation algorithm. At the end of this procedure, the composite vector is completely vectorized and ready for the next tick.

\subsubsection{Simulation Clearance Algorithm}
Simulation clearance is fairly straightforward since all that needs to be done is apply the \texttt{clear} function to all vectors and maps existing in simulated contexts. This is done because there should be no interference between ticks. All data before the end of a tick is transitioned out of the simulation anyway, so it is safe to clear without the risk of losing data.

For archetypes:
\begin{itemize}
    \item entt\_delete\_queue
    \item entt\_creation\_queue
    \item entt\_mutation\_queue
\end{itemize}

For simulated worlds:
\begin{itemize}
    \item entity\_registry
    \item archetype\_registry
    \item Everything in the simulated archetypes
\end{itemize}

\subsection{Entity FSM}
The entity FSM is a collection of internal functions in the \texttt{archetypes.c, gecs.c} files. The most important of which is the \texttt{delta\_transition} function, which given a world, an id, and a set of types, performs an entity state transition.

\subsubsection{Entity $\delta$ Transitions}
Archetype transitions roughly follow the algorithm defined in chapter \ref{chap:2}. The algorithim goes as follows:

\begin{figure}[htbp]
    \begin{enumerate}
        \item Given world $w$, entity ID $e \in w.\texttt{entity\_registry}$, and a set of types $t \in \mathcal{P}(w.type\_set)$
        \item Create the new archetype if this is the first time seeing set $t$
        \item Load $a \leftarrow$ next archetype
        \item If entity $e$ is $\emptyset$, perform Algorithm \ref{alg:empty_to_a} and exit
        \item Push an empty composite element into $a$
        \item Copy over specifically all types that were retained from the old archetype into the new archetype (perform $a.\texttt{composite} \leftarrow a.\texttt{type\_set} \cap old_a.\texttt{type\_set}$)
        \item Update entity record $e$ to have the last index in the \texttt{composite} and archetype pointer to $a$
        \item Update the old archetype's \texttt{dead\_fragments} member to contain $e$ since $e$ left the old archetype.
    \end{enumerate} 
    \caption{Entity General Transitions Algorithm}
    \label{alg:transition}
\end{figure}

\begin{figure}[H]
\begin{enumerate}
    \item Given archetype $a$, and entity $e$ who is in archetype $\emptyset$ 
    \item Push an empty fragment into composite of $a$,
    \item Update entity positions of a to include $e$ with last index in the \texttt{composite}
\end{enumerate}
\caption{Entity $\emptyset \rightarrow a$ Transition Algorithm}
\label{alg:empty_to_a}
\end{figure}

It's important to note that algorithms \ref{alg:transition} and \ref{alg:empty_to_a} are not thread safe but are still used in concurrent contexts. This is possible because simulations and the real context both contain the same members. So this algorithm is also used by the ledger to simulate the following entities positions.

\subsubsection{Concurrent Entity Component Additions and Access}
To add a new component while a system is running is actually remarkably simple. Since adding a new component guarantee's a $\delta$-transition, all we do is do a $\delta$ transition inside the simulated context. This is equivalent to the library user adding a component, the same function is used but its applied to the archetypes simulated world instead. 

Accessing is the same story as adding a new component. In the situation there exists the same component in the simulation and in the real context's currently processing archetype, GECS will always prefer the component in the real context. This is in order to preserve vectorizability and minimize moves out of the simulation.

\subsubsection{Concurrent Entity Component Deletions}
The deletion algorithm is a tricky one due to the use case of deleting components that existed when tick started. This means when deleting, a genuine transition is required because if this entity stayed at the same archetype, it risks giving access to undefined memory. To avoid this, when deleting original components off the entity, the entire entity is offloaded into the simulatiton and thats where the state transition occurs. 

Clearly, this makes a huge mess of things but it's nothing that cannot be recovered from. GECS uses a preference system when applying component access. The components on the archetype will stay fresh while the ones that were used to simulate the deletion transition inside the simulation will become stale. 

\begin{figure}[htbp]
    \begin{enumerate}
        \item Given a runtime query struct $q$, entity ID $e$, and set of types $t \in \mathcal{P}(q.\texttt{component\_registry})$
        \item Check if the simulation archetypes types of $e$ intersect with the entities current types. If they do not, this is the first time doing a deletion transition. Do the following steps for the first time, else skip to step 5:
        \item Construct a type union between the simulation and archetypes type set. 
        \item Transition with the type union into the simulation
        \item Apply the thread unsafe component removal function to the simulation and exit
\end{enumerate}
\caption{Entity Concurrent Component Removal}
\label{alg:conc_rem}
\end{figure}

\subsubsection{Entity Consolidation Algorithm}

The entity consolidation algorithm is used to clear the cache of mutations saved over the progress of a tick. These mutations exist in the buffers: entity creation, deletion, and mutation. This set of algorithms clears these three buffers and migrates all changes into the real context. Since GECS prefers to always use component data from the original archetype composite, there is a division between fresh components and stale components. The deletion operation is the only operation that produces these stale components. With this in mind, GECS performs the following algorithm to merge entities:

\begin{figure}[H]
    \begin{enumerate}
        \item Given the sequential world $w$
        \item For each archetype $a \in w.\texttt{archetype\_registry}$, perform the following steps:
        \item Perform Algorithm \ref{alg:entt_del_queue}
        \item Perform Algorithm \ref{alg:entt_cre_queue}
        \item Perform Algorithm \ref{alg:entt_merge}
\end{enumerate}
\caption{World Merge Algorithm}
\label{alg:world_merge}
\end{figure}

\begin{figure}[htbp]
    \begin{enumerate}
        \item While the delete queue is not empty:
        \item $e \leftarrow \texttt{pop()}$
        \item Remove $e$ from the the worlds member: \texttt{entity\_registry} 
        \item Remove $e$ from $e$'s archetype member: \texttt{entt\_positions} 
\end{enumerate}
\label{alg:entt_del_queue}
\caption{Consuming Entity Deletion Queue}
\end{figure}

\begin{figure}[htbp]
    \begin{enumerate}
        \item While the creation queue is not empty:
        \item $e \leftarrow \texttt{pop()}$
        \item If entity already exists in the worlds $\texttt{entity\_registry}$, continue
        \item Transition $e$ to be on the same archetype in the real context as it is in the $e$ simulated context
        \item Set all the components to have the same data as well
\end{enumerate}
\caption{Consuming The Entity Creation Queue}
\label{alg:entt_cre_queue}
\end{figure}

\begin{figure}[htbp]
\begin{enumerate}
    \item While the mutation queue is not empty:
    \item $e \leftarrow \texttt{pop()}$
    \item Load the simulated archetype of $e$ into $a\_{sim}$ and the real archetype of $e$ into $a\_{real}$
    \item Perform the type union between both archetypes into $x \leftarrow a_{sim} \cap a_{real}$
    \item Transition $e$ to archetype $x$
    \item Move over only new component data to $x$, ignoring types that already exist on $a_{real}$
\end{enumerate}
\caption{Entity Merge}
\label{alg:entt_merge}
\end{figure}

\subsection{Scheduling}
Scheduling is done the exact way as stated in chapter \ref{chap:2} and is no different in the implementation. Thus, this section is omitted.

\subsection{Support For Embarassingly Parallel Tasks}
GECS supports embarassingly parallel tasks my dividing dividing an archetypes composite into chunks to run on separate threads. All this is handled by GECS and the user only needs to pass in a function they deem thread safe.