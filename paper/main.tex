\documentclass[12pt,a4paper]{article}

% PAPER TITLE
\newcommand{\asgtitleshort}{Graph ECS}
\newcommand{\asgtitlelong}{A Graph Based Approach To Concurrent ECS Design}

% NAME
\newcommand{\nameshort}{Emil}
\newcommand{\namelong}{E.D Choparinov}

\usepackage[T1]{fontenc}
\usepackage{fontspec}
\defaultfontfeatures{Ligatures=TeX}

\usepackage{biblatex}
\bibliography{paper/refs.bib}

\usepackage{amsthm}
\usepackage{thmtools}
\usepackage{amsmath}
\usepackage{amsfonts}
\usepackage{amssymb}
\usepackage{textcomp,gensymb}
\usepackage{tikz}
\usetikzlibrary{arrows,automata}

\usepackage{algorithm}
\usepackage{algpseudocode}
\usepackage{minted}
\usepackage{xcolor}
\usepackage{csquotes}

\usemintedstyle{perldoc}

\usepackage{graphicx}
\usepackage{wrapfig}
\usepackage{caption}
\usepackage{subcaption}

\usepackage{hyperref}

\setlength{\textwidth}{16cm}
\setlength{\textheight}{22.5cm}
\setlength{\topmargin}{-2cm}
\setlength{\oddsidemargin}{0cm}
\setlength{\evensidemargin}{0cm}
\setlength{\parindent}{5mm}

\usepackage{color}
\definecolor{darkred}{rgb}{0.6,0.0,0.0}
\definecolor{darkgreen}{rgb}{0,0.50,0}
\definecolor{lightblue}{rgb}{0.0,0.42,0.91}
\definecolor{orange}{rgb}{0.99,0.48,0.13}
\definecolor{grass}{rgb}{0.18,0.80,0.18}
\definecolor{pink}{rgb}{0.97,0.15,0.45}

% listings
\usepackage{listings}

% Code definition
\lstset{
  aboveskip=1em,
  breaklines=true,
  abovecaptionskip=-6pt,
  captionpos=b,
  frame=single,
  numbers=left,
  numbersep=15pt,
  numberstyle=\small,
  showstringspaces=false
}


% Code color theme definition
\lstdefinestyle{colored}{ %
  basicstyle=\ttfamily,
  backgroundcolor=\color{white},
  commentstyle=\color{darkgreen},
  keywordstyle=\color{blue}\bfseries\itshape,
  stringstyle=\color{red},
  morekeywords={A,B,C}
}

% Theorem Defintions
\colorlet{shadecolor}{white}
\newtheorem{thm}{Theorem}

\newtheoremstyle{definitionsty}{15pt}{15pt}{\slshape}{}{\bfseries}{.}{.5em}{}
\theoremstyle{definitionsty}
\newtheorem{tdefn}{Definition}
\newenvironment{defn}
  {\begin{shaded}\begin{tdefn}}
  {\end{tdefn}\end{shaded}}

\usepackage{blindtext}

\usepackage{fancyhdr}
\pagestyle{fancy}
\fancyhf{}
\usepackage{lastpage}

\setlength{\headheight}{65pt}
\rhead{\large \asgtitlelong}
\lhead{\large \namelong}
\rfoot{Page \thepage /\pageref{LastPage}}

\newcommand{\ZZ}{\mathbb{Z}}
\newcommand{\RR}{\mathbb{R}}

\author{\namelong}
\title{\asgtitlelong}
\date{\today}

\begin{document}

\maketitle

\thispagestyle{fancy}
\begin{abstract}
  The Entity Component System, also known widely as ECS, is the most important architectural design pattern for realtime simulations. By decoupling data from logic, ECS architectures uniquely excel at faciliting compositions of dynamic objects. The objective of this thesis is split into four parts. 

  The study begines with an introduction and literature review of the ECS pattern, highlighting its strengths and weaknesses, and includes the motivation as to why to use an ECS over Object-Oriented Programming in realtime simulation contexts. The paper will then move on to a review of the implementation styles noticed while researching various real-world ECS implementations. Chapter \ref{chap:1} concludes with touching broadly on the concurrency styles those real-world ECS's used in their implementations.
  
  Chapter \ref{chap:2} is dedicated to the concurrency techniques used in the GECS library, the ECS implemented for this paper. The introduction begins by discussing the topic of vectorization and rethinking about components as types and archetypes. It then explores using archetypes to construct a runtime entity finite-state-machine (FSM) to aid in scheduling the ECS in concurrent contexts. Before wrapping up the chapter, it touches on how to retain validity of the entity FSM over runtime.
  
  Chapter \ref{chap:3} finally introduces the GECS library implemented for this paper. This chapter covers all topics related to implementation details to data layout and the structures used. Preserving the vectorizability of archetypes and the concurrency implementation are explained in greatest detail. The remainder of this paper is benchmarking and discussions of the results following from implementing GECS. 

  GECS aims to optimize for execution in general purpose scripting contexts from performing physics calculations to facilitating user interactions. It's important to note before reading futher on that due to the nature of this subject, scientific journals regarding this topic is limited. However, articles about designing ECS architectures and presentations are widely availabile online from reputable organizations.

  Experimental results are **TDB**
\end{abstract}

\newpage

\newpage
\tableofcontents
\newpage

\section{ECS Literature Review}
\label{chap:1}

Historically, the ECS design pattern was introduced in 1998 with the game titled "Thief: The Dark Project".\cite{RomeoPHD} And the 
motivation behind designing their ECS is that OOP techniques were plaguing the games industry at the time with massive class 
hiearchies, deep inheritance paths, and introduced highly specialized code to projects\cite{Haerkoenen2019}. 
The ECS pattern argues from the position that it can solve all these problems by limiting the hierachy to contain only three layers. It exists as an architectural pattern that follows Data-Oriented-Design (DOD) principles.\cite{RomeoPHD} To this day, the ECS pattern maintains its importance in simulations. In 2015, Apple introduced an implementation of ECS into their GameplayKit API framework. \cite{AppleECS}

Literature on the ECS pattern is limited and hard to find. Although, as mentioned in the abstract there are many articles, conferences, and well-documented libraries from respected individuals in the games industry. GECS is inspired by the articles written by Sander Mertens, the creator of the very popular FLECS framework and can be found in \cite{SanderMertensECS}. 

\subsection{Entity Component System Foundations}
These are the foundational definitions used in this paper. All are based from the FAQ provided by \cite{SanderMertensFAQ} and \cite{RomeoPHD} research.

\subsubsection{Entity}
    An entity is a unique identifier used to represent a "object" in a simulation. This unique identifier is used in some manner by the ECS to collect a set of components. Some ECS's try to make the identifier intelligent to optimize component selection bits in the identifier to store more location information. This will be discussed in the existing ECS implementions.

\subsubsection{Component}
    Generally speaking, a component is data. It can be defined as a struct, tuple, or class. For this thesis, a component is defined as a segment of either complete or incomplete data. This will be discussed in part \ref{chap:3}.

\subsubsection{System}
    A system is a function defined by the ECS user that intends to execute operations over a set of entities. It 
    contains two parts: the query and execution phase. A system queries for a collection and then executes a function 
    that modifies, creates, or destroys the objects of that selection.

\subsubsection{Tick}
    A tick is defined to be a serialization point during runtime in which all systems have been processed and the ECS is
    ready to process them all again. This is the serialization point used in the model.

\subsection{Motivation \& The Weaknessess Of Object Oriented Programming}

% TODO: Sources
It has been exhaustively shown that Object-Oriented Programming increases the difficulty of achieving desired simulation goals and that there are plenty attractive alternatives. The two main weaknesses that an ECS intends to solve is: poor cache utilization and poor parallelization.\cite{RomeoPHD}

Large projects deep into using OOP principles learn how hard it is to keep cached what you need and keep junk out. OOP principles can make pointer abuse very easy and cause the cache to overwrite potentially useful data in the near future. ECS solves this by introducing components. Since components are detached from their entity, we as engine designers can put these components wherever we please. It's not an uncommon practice to see component data be stored in long homogenous vectors. Suppose you have to do a heavy operation across the set of entities that all contain component $Q$. An ECS in this situation will be extremely cache friendly, due to $Q$ being next to components that all contain the same component type.

In OOP, parallelism can be painful due to synchronization errors over memory that was not engineered to effectively be
parallelized. Because ECS follows principles from Data-Oriented Design, an ECS can generally produce code that is more
parallelizable by default\cite{RomeoPHD}. The concept of storing components as homogeneous vectors not only helps with caching but because all data pertaining to a component is already homogeneous -- it's easy to process in parallel or in groups. \cite{Wiebusch2012}\cite{SanderMertensECS}

\subsection{Example Entity-Component System Model}
\label{sec:ecs_naive}
The following section discusses a naive implementation to give some understanding as to how to organize data together. This
naive implementation is not thread safe and is presented as C code.

\subsubsection{Structuring Data}

Below is the members of a hypothetical naive ECS implementation. This implementaion style uses the same style that GECS
uses, called Dense ECS \cite{EnTT_SparseSets}.

\begin{figure}[H]
\begin{lstlisting}[
    language=Java,
    numbers=none
]
struct system {
    void (*run_sys)(ecs *e); /* Fun : user defined sys function */
    map_t components;        /* Map : component -> *Vec : component */
};
struct entity_record {
    map_t component_table;   /* Map : component id -> index */
};
struct ecs {
    int64_t id_gen;          /* Int : generates unique entity ids */
    map_t component_table;   /* Map : component id -> Vec: component */
    map_t entity_table;      /* Map : entity id -> entity_record */
    vec_t systems;           /* Vec : system */
};
\end{lstlisting}
    \caption{Example ECS Members}
    \label{code:naive_ecs_data}
\end{figure}

ECS architecture are built upon two core data structures: vectors and maps. As seen above, it has all the parts an ECS requires to function. When initializaing, \texttt{id\_gen} is set to 0 and increments for each new entity created. The component table maps a given component ID to a vector of that type of component. This vector can then be returned immediately by the ECS to allow for a user to process all entity components of a specific kind with a linear operation and a huge win in cache locality.  

Already, there are many opportunities to parallelize.

\subsubsection{Accessing Single Entities}
In order to create a new entity the following steps are taken to manipulate the data within the structs:

\begin{enumerate}
    \item Load x \(\leftarrow\) \texttt{struct ecs}
    \item Increment \texttt{x.id\_gen}
    \item Return \texttt{x.id\_gen}
\end{enumerate}

In order to perform add, create, modify, or delete operations off of a specific entities component data the \texttt{entity\_record} must be loaded from the \texttt{entity\_table}.

\begin{enumerate}
    \item Load \texttt{x} \(\leftarrow\) \texttt{struct ecs}
    \item Load \texttt{r} \(\leftarrow\) \texttt{x.entt\_table}
    \item Load \texttt{c} \(\leftarrow\) \texttt{r.component\_table}
    \item Perform operation on \texttt{c}
\end{enumerate}

As can be seen in the two algorithms above, a Dense ECS is not for modifying a single entity at a time but many all at once. So, single entity accesses are intentionally slower that component accesses. \cite{EnTT_SparseSets}

\subsubsection{Accessing Components}
A core feature of an ECS implementation is for the ability to query systems based on what components the entities inside the system require. Based on that, each system when registered with the ECS will collect pointers to which component vectors they
are capable of working on. 

When registering a system:
\begin{enumerate}
    \item For each component ID \( \texttt{x} \leftarrow [\ldots] \)
    \item Load ptr \( \texttt{p} \leftarrow \texttt{struct ecs : entity\_table} \) such that \texttt{p} contains the vector for \texttt{x}
    \item Save $(\texttt{x}, \texttt{p}) \rightarrow \texttt{struct system : components}$
\end{enumerate}


When registering a component, notice how there is no component id generator. This implementation offloads the task of id
generation to C having unique types. Using macros, the type name can be sent as a stack allocated string to a function to be digested into a component id. The hashing function used by GECS and this model is djb2 \cite{hashing}, a nice and fast hashing function that has decently low collision probability.

% https://github.com/SanderMertens/ecs-faq
\subsection{Existing ECS And Implementation Styles}
From part \ref{chap:1}, its easy to surmise that the ECS pattern can be implemented in many different ways, giving way to unique performance benefits and tradeoffs per style. This section covers an overview of various implementation styles discovered around ECS architectures.

\subsubsection{Entity ID Styles}
A lot can be done with just reserving a couple bits on IDs generated. The popular ECS framework called FLECS, short for Fast 
Lightweight Entity Component System, proposes to represent all component ID's generated in the same manner entity ID's are generated. Notice how in section \ref{sec:ecs_naive}, the naive ECS contains two different methods of ID generation: hashing component typenames and an ID generator counting up. FLECS splits a \texttt{uint64\_t} into two parts where the lower 32 bits represents the entity and the upper 32 bits are reserved for optimizing internal ECS mechanisms.

\begin{figure}[H]
    \centering
    \includegraphics[width=0.5\linewidth]{resources/entity_generation.png}
    \caption{Smart Entity Generation in FLECS}
    \label{fig:entity_generation}
\end{figure}

\textbf{Runtime Tagging:}
A tag in an ECS is a component with no data. Tags are used generally to apply entities to systems without the intent of editing any data based on the tag, but the components adjacent to it. Since components now generate the same way entities
do, entities can now share a direct relationship to a component via runtime tagging as shown in the example below. \cite{RomeoPHD}

\begin{figure}[H]
    \begin{lstlisting}[
        language=Java
    ]
component CloseCircle = world.component();
entity John = world.entity();
entity Mary = world.entity();
entity Brad = world.entity();
CloseCircle.add(John, Mary, Brad);
CloseCircle.each([](entity friend) { /* John, Mary, and Brad */ });
\end{lstlisting}
    \caption{Runtime Tagging Example}
    \label{code:runtime_tagging}
\end{figure}

The code in figure \ref{code:runtime_tagging} on line 5 is when the ID of \texttt{CloseCircle} is modified to contain some direct relationship to the entities $[\texttt{John}, \texttt{Mary}, \texttt{Brad}]$.

\textbf{Reflection:}
The upper 32 bits can be used for reflection. Reflection is the ability to serialize component data out of the ECS. Since entities are also components, its simple to serialize. Serialization can be done by constructing a new component to house components as if it were an entity, similar to the code in Figure \ref{code:runtime_tagging} but if all the entities were components. By doing this, reflection can be achieved with minimal effort. All that is required needed is to iterate over the components reflected as if they were entities.

\textbf{Entity Liveness:}
A core issue with the ECS pattern is that many games will burn through ID's fast. As such, recycling IDs is important part of an ECS implementation. FLECs reserves the first 16 bits of the upper 32 bits of entity identifiers as a generation count. So a counter for the counter of how many times this ID has been recycled. This allows for easy checking if an entity is still alive.  

\textbf{Entity Relationships:}
One of the biggest flaws of ECS architectures is that perfoming direct entity relationship queries is slow and not performant because of how components and entities are organized. By adding relationships to other entities in the ID, query times based become more efficient. For more information on how this is done inside FLECS, which pioneered this techinque, read Sander Merterns article in \cite{SanderMertensEntityIDs}.

\subsubsection{Dense ECS}
\begin{figure}[htbp]
    \centering
    \includegraphics[width=0.5\linewidth]{resources/dense_ecs.png}
    \caption{Dense ECS Architecture Example}
    \label{fig:dense_ecs}
\end{figure}

Dense ECS's, also known as Table based ECS's, store entities inside of large tables. In these tables, components and entities are columns and rows respectively. This gives way to fast component or entity queries, depending on which is is contiguous in memory. Some popular types of this ECS is FLECs or the Unity Game Engine's own ECS, Unity DOTS. \cite{SanderMertensFAQ}

\subsubsection{Sparse ECS}
To understand Sparse ECS architectures, a small introduction into an exotic data structure is necessary. A Sparse Set is a datastructure dedicated to keeping the invariance of the following property:
\begin{equation*}
    \forall v \in \{0,\ldots, n-1\} : \text{D}[S[v]] = v
\end{equation*}
In the property above, $D$ and $S$ represent two vectors. The dense vector and sparse vector respectively. By using these two vectors in unison: lookup, insertion, and deletion time are all $O(1)$ and iteration through the set is $O(n)$. Interestingly enough, this data structure is one that does not require memory to be initialized. \cite{sparse_profit}. The only downside to Sparse Sets is that they must be capable of representing an entire domain since direct index access is used. If, for example $v = 10000$, then there must be at least $10000$ elements in $S$. $D$, the dense vector, will always be the length of the set itself.

The main advantage of sparse ECS implementations is how fast it is to test if an entity contains a component. Because of this, sparse ECS's advantages come in the form of query optimization. In a sparse ECS, a query collects entities by roughly performing an algorithm that queries for all entities that is the union of true values returned from all component sparse sets. Bitset ECS's are known to use similar approaches in their implementations.\cite{EnTT_SparseSets}

There is an alternative use with sparse sets in ECS architectural contexts. Systems could maintain their own sparse sets, which are kept up to date before each tick starts. These sparse sets are used by systems to collect entities related to the system before the system runs itself each tick. An example of an ECS that does this technique is ecst.

Sparse sets also have another use where systems own their own sparse sets and they are kept to date each tick during runtime. These system owned sparse sets are then used to collect entities related to the system. An example of an ECS that does this technique is ecst. \cite{ecst}

Overall, it's important to know that sparse set ECS's are popular in situations where fast add/remove of single component operations are preferred over efficient batch operations. To back this claim, FLECS which is a popular dense ECS published its own benchmarking data against other industry-wide used ECS's. From their data, FLECS has slower add/remove component operations than EnTT, a popular sparse set based ECS.\cite{FLECS_EnTTCompare} Regardless of the advantages and disadvantages of using sparse vs dense ECS's, it can be summed up nicely by the creator of EnTT \cite{EnTT_archetype_and_quote}:

\begin{quote}
    \textbf{It’s a matter of taste}, in fact.
        - Michele Caini
\end{quote}

\subsubsection{Bitset-based}
Another exotic structure used in ECS implementations are bitsets. A bitset is analogous to a hashset. It tracks if certain indices exist are within a set, but the implementations differ vastly however. Bitsets are known for their memory efficiency and are preferred over hashsets if the set is known to store large volumes of objects in the set. \cite{Sutherland2014}

A bitset-based ECS stores components in contiguous arrays where the entity id is used to index. The bitset is used in this application to check if the entity contains that component. There are many different flavours of approaches using this style, such as using a hierarchical bitset structures. \cite{SanderMertensFAQ} EntityX is one of the architectures reviewed that uses bitsets.

\subsubsection{Reactive}
A reactive ECS is more different than the others in this list. A reactive ECS uses signals emitted from mutating entities to keep individual lists of which entities belong to which system. Entitas is an example of this. \cite{SanderMertensECS}

% \begin{verbatim}
%     Entitas ECS

% +-----------------+
% |     Context     |
% |-----------------|
% |    e       e    |      +-----------+
% |       e      e--|----> |  Entity   |
% |  e        e     |      |-----------|
% |     e  e     e  |      | Component |
% | e          e    |      |           |      +-----------+
% |    e     e      |      | Component-|----> | Component |
% |  e    e    e    |      |           |      |-----------|
% |    e    e     e |      | Component |      |   Data    |
% +-----------------+      +-----------+      +-----------+
%   |
%   |
%   |     +-------------+  Groups:
%   |     |      e      |  Subsets of entities in the context
%   |     |   e     e   |  for blazing fast querying
%   +---> |        +------------+
%         |     e  |    |       |
%         |  e     | e  |  e    |
%         +--------|----+    e  |
%                  |     e      |
%                  |  e     e   |
%                  +------------+
% \end{verbatim}

\subsection{Concurrency in ECS Implementations}
The majority of ECS implementations investigated had concurrency support but were severely undocumented. So only two were investigated: FLECS and EnTT.

\subsubsection{FLECS}
The concurrency model of FLECS is simple. The user initially sets the amount of worker threads manually, and entities matched with a system will be divided equally accross threads. Changes to entities are stored in what Sander Mertens calls a "command queue" that are later merged before the end of the tick.\cite{missing_docs} This system is similar to the concurrency model I used for GECS most likely because both GECS and FLECS are Dense ECS's that rely on vectorizing type compositions for concurrency gains.

\subsubsection{EnTT}
The EnTT framework does not do anything special to for concurrent performance gains since their design naturally allows for parallelizable code to be threaded by the user themselves. Michele Caini states in his documentation that "Views, groups, and interators in general aren't thread safe by themselves" but then gives suggestions on how to apply concurrency using their architecture. The author also ends on a note saying that they at least support concurrent entity generation by setting the \texttt{ENTT\_USE\_ATOMIC} flag.\cite{EnTT_multithreading}
% TODO: maybe add another section about how vectorization actually works in archetypes vs types?

\section{Concurrency Model}
\label{chap:2}
The following concurrency model is of my own design. To the best of my ability I was unable to find a model like the one presented in this paper other than FLECS to a degree. Since our implementations follow similar techniques such as using archetypes and being a dense ECS, our concurrency models seem to overlap slightly but further investigation is needed.

\subsection{Formal definition} \label{section:formal_definition}
In this paper, we define an Entity Component System, short for ECS, to be a tuple $W = (T, A, \delta, \Lambda)$ consisting of:
\begin{enumerate}
    \item A finite set of types $T$
    \item A set of archetypes $A \subseteq \mathcal{P}(T)$, where $\mathcal{P}(T)$ denotes the power set of $T$    
    \item Some transition function $\delta : A \times T \rightarrow A$
    \item A set of systems $(\lambda, \lambda_{req}) \in \Lambda$ such that $\lambda : W \rightarrow W$ and $\lambda_{req} \subseteq \mathcal{P}(T)$ representing the types required to initiate $\lambda$
\end{enumerate}
This definition will be used in this chapter to generate the archetype graph.

\subsection{Types And Archetypes}
Until now, components were only stored in homogenous vectors which is great for cache locality and vectorizability but it introduces some problems -- namely with composite entities. We define a type $T$ to be the reference to a class of components. Therefore, a homogenous vector contains only components of type $T$.

\subsection{Vectorization}
Vectorization is a kind of data parallelism such that an operation is simultaneously applied to different segments of data. This is typically achieved via SIMD. Regardless, for the purpose of an ECS, vectorization means that the data needs to follow a strict standard of being contiguous in some manner. This is because of the convention that a system applies changes over components more than entities. By having archetypes be made out of components, and archetypes are able to generate a runtime Entity FSM, vectorization at the archetype level is possible. 

\subsubsection{The ABC Problem}
% https://ajmmertens.medium.com/building-an-ecs-2-archetypes-and-vectorization-fe21690805f9
The ABC Problem is a problem introduced by the FLECS creator Sander Mertens. It's a problem that demonstrates the effectiveness of Archetypes in component storage, it retains a vectorizable property, and is a good introduction to archetypes. 

Outside of ECS world $W$, suppose there exists a set of entites $E$ and $n = |E|$. Suppose $T := \{A,B,C\}$ and for each type $t \in T$, there exists a homogenous vector containing elements of type $t$. If all entities have all the components then:

\begin{figure}[H]
    \begin{lstlisting}[
        language=Java,
        numbers=none,
        style=colored
    ]
    A components[n];
    B components[n];
    C components[n];
    \end{lstlisting}
    \caption{Homogenous ECS Components}
    \label{code:homogenous_ecs}
\end{figure}

In this ideal world, entity ID can be emulated by the index to all vectors because the property $\forall t \in T : |t| = n$ holds. This ECS is positioned advantageously because all vectors are contiguous and therefore are vectorizable. This is only true because of the property mentioned.

Suppose one of the entities at some index $k$ removes some component $t$. In such a situation, the defined indexing property is broken because not all vectors are of the same length and the gap in memory now means it is impossible to write vectorized code for this ECS.

Sander Mertens uses this setup to prove that an ECS cannot vectorize homogenous components efficiently at all. This is done by reviewing all situations in which homogenous components are mutated to make vectorization impossible or difficult. By the end, he presents an alternative called archetypes and proves that archetypes leads to some form of good vectorizability.

\subsubsection{Archetypes}
Simply put, an archetype is a set of types as defined in section \ref{section:formal_definition}. Archetypes add a layer of abstraction on top of types so to make ECS queries vectorizable. Instead of creating homogenous vectors based on types in $T$, we can create homogenous vectors based on archetypes in $A$. 

\begin{figure}[H]
    \begin{lstlisting}[
        language=Java,
        numbers=none,
        style=colored
    ]
    // Type {A} "A"
    A a[A_len];
    // Type {A, B} "AB"
    A a[AB_len];
    B b[AB_len];
    // Type {A, C} "AC"
    A a[AC_len];
    C c[AC_len];
    \end{lstlisting}
    \caption{ECS Components With Archetypes $\{\{A\},\{A,B\},\{A,C\}\}$}
    \label{code:ecs_archetypes}
\end{figure}

Even though in Figure \ref{code:ecs_archetypes} those arrays are independent they do not have to be. In my implementation, for example, all components in an archetype are interleaved in a contiguous array. So essentially the papers archetypes storage looks more similar to Figure \ref{code:homogenous_ecs} than Figure \ref{code:ecs_archetypes}.

Note that since archetypes are sets of types, the archetypes $AB$ and $BA$ both represent the same archetype.

\subsection{Representing Archetypes As An Entity Finite-State-Machine}
\label{sec:fsm_arc}
ECS applications are expected to handle operations over entities that can add, modify, or remove components. In the context of archetypes, this means being able to transition an entity between archetypes. 

An example of such a situation is a game. Suppose all entities within the proximity of some point in space gain a tag component saying "buffed", but when they leave the proximity the tag component is removed. These types of state transitions may occur hundreds of times and need to be performant. 

\subsubsection{Archetype Graphs}

A finite state machine emerges from using archetypes to organize entities in a specific way. If the addition and removal of components are considered as state transitions, then the following in Figure \ref{fig:graph1} represents existing archetypes during some ECS runtime. We define vertices to be archetypes in $W$ and edges to be component addition or removal transitions to adjacent archetypes.

As an example, suppose we have an ECS with the following properties:
\begin{align}
    T &= \{A,B,C,D\} \\
    A &= \{ \emptyset, [A] , [A,C] ,[B], [A,B], [A,B,C], [B,D], [D]\} \\
    \Lambda &= \emptyset
\end{align}

\begin{figure}[htbp]
    \centering
    \includegraphics[width=0.5\linewidth]{resources/graph1.png}
    \caption{Example FSM Graph Of Archetypes During Runtime}
    \label{fig:graph1}
\end{figure}

As shown in Figure \ref{fig:graph1}, all graphs contain the empty set $\emptyset$ as a vertex. This is because all entities when initiated contain the the archetype with no components. As components get added to entities over their runtime, entities perform state transitions to their designated archetype. 

Archetype graphs display the following properties:

\begin{enumerate}
    \item All existing archetypes must have a path to vertex $\emptyset$.
    \item Components can appear multiple times.
    \item Archetypes can only appear once.
\end{enumerate}

There are a couple interesting things to notice in Figure \ref{fig:graph1}. Notice how there is no transition between state $[B,D]$ and $[D]$. This is because this graph represents a runtime of an ECS and while it is possible to transition between $[B,D] \leftrightarrow [D]$, it has yet to happen in this runtime. 

The component $C$ in Figure \ref{fig:graph1} is special in the regard that it is not adjacent to the $\emptyset$ vertex. This does not mean that it is impossible to create an entity with only component $C$ but that the runtime has yet to have that situation occur.

To loop back to the proximity example in section \ref{sec:fsm_arc}, by representing archetypes as runtime state transitions we can now see that large performance gains are achieved via these edges. Initially, the first entity that transitions between two states in the simulation will be slow because the edge between those two vertices must be made. Once that edge is made, it is cached and the deletion and transfer of an entity to another archetype drops from $O(N)$ to $O(1)$.

\subsubsection{Opertions On Entity FSM}
The following operations are all thread safe due to the scheduling algorithm explained in section \ref{sec:scheduling}. All implementation details are abstracted away except the operations directly done on the graph. 

\textbf{Entity Creation:} All entities when created start at the $\emptyset$ vertex. All entities that exist here are not queryable. In the papers implementation, the $\emptyset$ vertex is used as a garbage collection tool. All entities waiting for deletion are marked to transition to the empty set and cleaned at the end of the tick. 

\textbf{Entity Component Addition: } Aside from the simple transition using an existing edge. When a component addition occurs, two operations are completed. Suppose the archetype we are currently in is at archetype $G$ and are attempting to add component $C$ to transition to $G \cup \{C\}$. The two operations presented are the lookup operation and the connection operation.

\begin{enumerate}
    \item If $G \cup \{C\} \not\in A$, then do step 2. Else do step 3.
    \item Create a new archetype $G \cup \{C\} \in A$.
    \item Create the edge $G \leftrightarrow G \cup \{C\}$.
\end{enumerate}

When considering concurrency, data invalidation can occur if the archetype to transition to exists and another thread or system is using that archetype. In such a situation, the papers implemention waits for that part of the graph to be free'd.

\textbf{Entity Component Deletion: } When an entity is marked to have a component deleted. The same algorithm is used in entity component addition. Both addition and deletion use the same transition function $\delta$ of the papers ECS definition.  

\textbf{Entity Deletion: } When an entity is marked for complete deletion, the archetype it was part of must have at least one more entity to exist. If the archetype is empty, it must be removed from the graph. While this property keeps the graph mathematically pretty, there is no reason to programatically remove cached edges as they have a probability to appear again. The memory usage of an empty archetype is inconsequential. 

\subsection{Scheduling Systems}
\label{sec:scheduling}

Due to the nature of ECS patterns following some principles of data-oriented design, these architectures have an advantage in concurrency contexts. The following section is about an algorithm designed in this paper to schedule what is parallelizable and what is not parallelizable. This algorithm does not guarentee that all entities will run at the same time, but does guarentee that all entites will have been processed once per tick at some point in time. Such is the nature of concurrency. 

\subsubsection{Syntax Introduction}
For the following algorithms in section \ref{sec:scheduling}:

\textbf{Partial Systems:} A partial system $\lambda^\prime$ is defined as a system that processes a specific set of archetypes, instead of its requirements. The following is the syntax definition:

\begin{equation}
    \lambda^\prime := \{a_1,a_2,\ldots,a_n\} \in \mathcal{P}(A) : \lambda[a_1][a_2]\ldots[a_n]
    \label{eq:partial_lambda}
\end{equation}

\textbf{Set Symmetric Difference:} $\Delta$ used in the following algorithms is defined as the symmetric difference: 
$$x_1 \Delta x_2 := (x \cup x_1) \setminus (x \cap x_1)$$

\subsubsection{Partitioning Algorithm}
\label{alg:part}
The following algorithm presented in the paper is used to determine what systems are parallelizable based on the entity FSM introduced in section \ref{sec:fsm_arc}. 

\begin{enumerate}
    \item Initiate sets $par, seq$ such that they will contain tuples of type $(\lambda^\prime)$
    \item Initiate vector $v_1$ that will contain tuples of the type $(\lambda \in \Lambda, x \subseteq \mathcal{P}(A))$
    \item For each $\lambda \in \Lambda$, do the following steps 4-5:
    \item Construct the set $x = \{ \forall a \in A : a \cap \lambda_{req} \not= \emptyset \}$
    \item Push to $v_1$: $(\lambda, x)$
    \item For each $(\lambda, x) \in v_1$, do the following steps:
    \item Calculate the sequential set $X_{seq} = \{ (\lambda_1, x_1 \cap x ) : (\lambda_1, x_1) \in v_1 \land \lambda_1 \not= \lambda \}$. This set contains all archetypes that appear in $\lambda_1$ relative to $\lambda$.
    \item Calculate the parallel set $X_{par} = \{ (\lambda_1, x_1 \setminus x ) : (\lambda_1, x_1) \in v_1 \land \lambda_1 \not= \lambda \}$. This set contains archetypes that are parallelizable relative to $\lambda$.
    \item Push to set $seq$ the value $ \forall x_0,x_1,\ldots,x_n \in X_{seq} : \lambda[x_0][x_1]\ldots[x_n]$
    \item Push to set $par$ the value $ \forall x_0,x_1,\ldots,x_n \in X_{par} : \lambda[x_0][x_1]\ldots[x_n]$
\end{enumerate}

After all the steps above, the set $seq$ contains $\lambda^\prime$ that are archetype dependent on archetypes that appear in other $\lambda$'s that do not derive $\lambda^\prime$. This set is still partially parallelizable as will be shown in the scheduling algorithm and computation graph.

Although the algorithm for partitioning in section \ref{alg:part} is $O(N^2)$ where $N$ is the number of archetypes, its unlikely that there will ever enough types of components and lazily loaded archetypes on the graph for it to matter in any circumstance, but it is important to still bring awareness.

% TODO: ACTUALLY DO IT
Depending on the leniency of defining $\lambda_{req}$, its possible to only ever have to do this calculation once. If all systems are defined at compile time and new systems are not allowed to be added during runtime, the computation graph will never change. The implementation in this paper allows new systems to be defined at runtime and only needs to recalculate on new archetype additions to the Entity FSM.

\subsubsection{Example Computation}
The following section is dedicated to producing the $seq$ and $par$ sets based on the entity FSM in Figure \ref{fig:graph1}. Suppose instead of $\Lambda = \emptyset$:

\begin{equation*}
\Lambda = \begin{cases}
    \lambda_1 &= [A,C] \\
    \lambda_2 &= [B,C] \\ 
    \lambda_3 &= [D]    
\end{cases}
\end{equation*}


\textbf{Vector Production:} The first step is generating the $v_1$ vector.

\begin{align*}
    v_1[\lambda_1] &= \{ \forall a \in A : a \cup [A,C] \neq \emptyset \} & \Rightarrow &
    \quad v_1[\lambda_1] = \{[A],[AC],[AB],[ABC]\} \\
    v_1[\lambda_2] &= \{ \forall a \in A : a \cup [B,C] \neq \emptyset \} & \Rightarrow &
    \quad v_1[\lambda_2] = \{[B],[AB],[AC],[BD],[ABC]\} \\
    v_1[\lambda_3] &= \{ \forall a \in A : a \cup [D] \neq \emptyset \} & \Rightarrow &
    \quad v_1[\lambda_3] = \{[D],[BD]\}
\end{align*}

\textbf{Sequential Set:} The next step is to generate the set $X_{seq}$ for each set in $v_1$.

\begin{align*}
    \lambda_1: \quad & X_{seq} = \{v_1[\lambda_1] \cap v_1[\lambda_2], v_1[\lambda_1] \cap v_1[\lambda_3]\} & \Rightarrow & \quad X_{seq} = \{[AC][AB][ABC],[]\} \\ 
    \lambda_2: \quad & X_{seq} = \{v_1[\lambda_2] \cap v_1[\lambda_1], v_1[\lambda_2] \cap v_1[\lambda_3]\} & \Rightarrow & \quad X_{seq} = \{[AC][AB][ABC],[BD]\} \\ 
    \lambda_3: \quad & X_{seq} = \{v_1[\lambda_3] \cap v_1[\lambda_1], v_1[\lambda_3] \cap v_1[\lambda_2]\} & \Rightarrow & \quad X_{seq} = \{[],[BD]\} \\ 
\end{align*}

\textbf{Parallel Set:} With $v_1$ and $X_{seq}$ generated, apply to each context of $X_{seq}$ above the formula to generate $X_{par}$.
\begin{align*}
    \lambda_1: \quad & X_{par} = \{v_1[\lambda_1] \setminus v_1[\lambda_2], v_1[\lambda_1] \setminus v_1[\lambda_3]\} & \Rightarrow & \quad X_{par} = \{[A],[A][AC][AB][ABC]\} \\ 
    \lambda_2: \quad & X_{par} = \{v_1[\lambda_2] \setminus v_1[\lambda_1], v_1[\lambda_2] \setminus v_1[\lambda_3]\} & \Rightarrow & \quad X_{par} = \{[AC][AB][ABC],[BD]\} \\ 
    \lambda_3: \quad & X_{par} = \{v_1[\lambda_3] \setminus v_1[\lambda_1], v_1[\lambda_3] \setminus v_1[\lambda_2]\} & \Rightarrow & \quad X_{par} = \{[],[BD]\} \\ 
\end{align*}

\textbf{Construct Sequential Partial Functions}: With the $X_{seq}$ generated, apply the transformation into $seq$:
\begin{align*}
    \lambda_1: \quad & X_{\text{seq}}[0]  \Rightarrow \quad \lambda_1^\prime[AC][AB][ABC] \\
                     & X_{\text{seq}}[1]  \Rightarrow \quad \lambda_1^\prime[] \\ 
    \\[0.3em]
    \lambda_2: \quad & X_{\text{seq}}[0]  \Rightarrow \quad \lambda_2^\prime[AC][AB][ABC] \\
                     & X_{\text{seq}}[1]  \Rightarrow \quad \lambda_2^\prime[BD] \\
    \\[0.3em]
    \lambda_3: \quad & X_{\text{seq}}[0]  \Rightarrow \quad \lambda_3^\prime[] \\
                     & X_{\text{seq}}[1]  \Rightarrow \quad \lambda_3^\prime[BD]
\end{align*}

Note how $\lambda_1^\prime$ and $\lambda_3^\prime$ contain partials with empty archetypes. Although these archetypes are valid, since $\emptyset \in \mathcal{P}(T)$, they do nothing so implementation wise they are unimportant. Using this computation, the following is the set $seq$.

$$
\text{seq} = \left\{
\begin{array}{@{}l@{}}
    \lambda_1^\prime[\text{AC}][\text{AB}][\text{ABC}] \\ 
    \lambda_2^\prime[\text{AC}][\text{AB}][\text{ABC}], \lambda_2^\prime[\text{BD}] \\
    \lambda_3^\prime[\text{BD}]
\end{array}
\right\}
$$

\textbf{Construct Parallel Partial Functions}: With the $X_{par}$ generated, apply the transformation to $par$:
\begin{align*}
    \lambda_1: \quad & X_{\text{par}}[0]  \Rightarrow \quad \lambda_1^\prime[A] \\
                     & X_{\text{par}}[1]  \Rightarrow \quad \lambda_1^\prime[A][AC][AB][ABC] \\ 
    \\[0.3em]
    \lambda_2: \quad & X_{\text{par}}[0]  \Rightarrow \quad \lambda_2^\prime[B] \\
                     & X_{\text{par}}[1]  \Rightarrow \quad \lambda_2^\prime[B][AB][ABC][AC] \\
    \\[0.3em]
    \lambda_3: \quad & X_{\text{par}}[0]  \Rightarrow \quad \lambda_3^\prime[D][BD] \\
                     & X_{\text{par}}[1]  \Rightarrow \quad \lambda_3^\prime[D]
\end{align*}

Note how $|\lambda_1:X_{seq}[1]| = |v_1[\lambda_1]|$. The fact that they are the same length means to us that with respect to $\lambda_3$, $\lambda_1$ is completely parallelizable. Finally, the set $par$ is:

$$
\text{par} = \left\{
\begin{array}{@{}l@{}}
    \lambda_1^\prime[\text{A}] \quad \lambda_1^\prime[\text{A}][\text{AC}][\text{AB}][\text{ABC}] \\ 
    \lambda_2^\prime[\text{B}] \quad \lambda_2^\prime[\text{B}][\text{AB}][\text{ABC}][\text{AC}] \\
    \lambda_3^\prime[\text{D}] \quad \lambda_3^\prime[\text{D}][\text{BD}]
\end{array}
\right\}
$$

\subsubsection{Scheduling Algorithm}
The two sets from the partitioning algorithm is all we need to schedule one tick.


\subsubsection{Computation Graphs}
Now with $par$ and $seq$, we are able to generate the computation graph for the systems. We sequentially process all partials in $seq$ first, and then spin up threads for each system. 

\begin{figure}[H]
    \centering
    \includegraphics[width=0.5\linewidth]{resources/computation_graph.png}
    \caption{Temp Graph To Replace Later with Latex}
    \label{fig:temp1}
\end{figure}

There is a visual way to understand this algorithm. A BFS can be initiated from all starting types from the given $\lambda_{req}$ to capture the vertices in $v_1$. Then we color each vertex set with unique colors. 

\begin{figure}[H]
    \centering
    \includegraphics[width=0.5\linewidth]{resources/color_graph.png}
    \caption{Temp Graph To Replace Later with Latex}
    \label{fig:graph2}
\end{figure}

In Figure \ref{fig:graph2}, notice that all nonoverlapping colors are safe to multithread while all overlapped colors ended up in the sequential part of the computation graph.

\subsubsection{Scheduling Systems}
Now to finally scheduling the systems. Because of the hard work done in section \ref{alg:part}, the set $seq$ can just run sequentially so that will be ignored and processed on the main thread. Once those are done, we assign for each partial system $\lambda^\prime \in par$ to the next available thread in the thread pool. 

\subsection{Entity Consolidation}
A key concept omitted thus far is how entities interact with the Entity FSM in concurrent contexts. There are three things in which an entity can do:
\begin{enumerate}
    \item Add a component and transition to a vertex not dependent on other concurrently running systems. 
    \item Add a component and transition to a vertex dependent on other concurrently running systems. 
    \item Move to $\emptyset$.
\end{enumerate}

\subsubsection{$\delta$ Transition To Non-Dependent System}
This case handled by the partitioning algorithm. Each entity, while running in the concurrent context, cannot appear twice in the same concurrent context. Entities can appear multiple times but never more than once while in the concurrent context. This is because archetypes organize entities and archetypes are what defines a partial system. 

\subsubsection{$\delta$ Transition To Dependent System}
% TODO: ACTUALLY DO THIS
This case can be handled in various ways. The easiest way is to add a constraint to $\lambda$ stating that operations such as these are invalid and must be included in the $\lambda_{req}$ set. The method this papers implementation chose is to have a local ledger of conflicting changes stored at the starting vertex. By doing this, we ensure that all concurrent systems finish with the same set of entities they deterministically started out on. Before the end of each tick, the ledgers of all vertices will be wipes and all entities with their accompanying data will be transfered to where they belong.

The ledger is defined as a map between entity ids and a queue of archetypes. If the entity wishes to transition to an archetype that is running concurrently, the entity will push the archetype to the queue. The full algorithm is detailed below:


\textbf{On Component Add:}
\begin{enumerate}
    \item An entity $e$ has requested for it to move from archetype $A$ to archetype $B$.
    \item Check if entity exists in the ledger. If so, go to step 3. Else, skip.
    \item Enqueue to entity $e$ archetype $B$. Done.
    \item Check if archetype $B$ belongs to another concurrent system. If so, go to step 3. Else, skip.
    \item Transition entity $e$'s archetype to $B$. Done.
\end{enumerate}

\textbf{On Archetype Move:}
\begin{enumerate}
    \item For each $a \in A$, do the following steps:
    \item For each $(e, q) \in a$ where $e$ is the entity and $q$ is the queue of types to apply.
    \item While $q$ is not empty, dequeue into $q^\prime$. If empty, Go to step 6.
    \item Transition entity $e$'s archetype to $q^\prime$.
    \item Go to step 2.
    \item Done.
\end{enumerate}

It's important to note that in such scenarios it's not only the new archetype that get's queue'd but the new component as well. The user must be able to dynamically construct and use these components while they are on the ledger unmoved. My implementation handles this.

\subsubsection{$\delta$ Transition To $\emptyset$}
This case is the deletion case and is handled using the same ledger in case 2. We push onto the ledger $(e \in E, \emptyset)$. Before the end of the tick, we iterate and clean out all entities that are supposed to transition to $\emptyset$. All that needs to be done is iterate through the ledger, and delete those entities. 

\subsection{Parallel Processing Internal Archetypes}
There are many cases in realtime interactive systems where data is embarassingly parallel. It would be nice if there was a way for us to support embarassingly parallel operations within the ECS. An common example of an embarassingly parallel system in an ECS is applying gravity to a set of objects. Calculating the next position of an entity is an isolated operation than can be divided up to threads since there is zero data dependencies. Luckily for us, archetypes are setup to be vectorizable and requires no need of synchronizations. Therefore we can just chunk out the vector evenly to threads that are available in the thread pool.
\section{GECS - Graph Entity Component System}
\label{chap:3}

The following chapter introduces the GECS C Library, built with the theory introduced in chapter \ref{chap:2}. Not all code will be presented in this paper, but the implementation is freely available online \href{https://github.com/EmilChoparinov/GECS}{here}. All code presented is written by me except for the hashing function and contains zero dependencies and is C99 compatible.

\subsection{Inspiration}
Before diving into the implementation, it's important to know how the theory connects with this library. GECS stands for Graph Entity Component System so obviously the inspiration is graphs. 

The graph part comes from the fact that the overall structure of the ECS is recursive. Instead of holding a ledger, each archetype simulates where entities are supposed to transition to instead. This means that on the Entity FSM, each archetype contains their own Entity FSM. The "real" world context holds the true Entity FSM while each other is a "simulated" world context. Roughly, at each tick the ledgers on each archetype gets their non-concurrent data dumped, transitioned, and ready for the next tick.

As such, GECS creates visual divisions in structures on what can be parallelized and what can't. The graph GECS is modeled from is shown in Figure \ref{fig:gecs_fsm}.

\begin{figure}[H]
    \centering
    \begin{verbatim}                                                                            
                       +------+    +------+     +---- Empty                                 
                       | +--+ |    | +--+ |     |     FSM                           
                       | |{}| |    | |{}| |<----+                                     
                    +->| +--+ | +->| +--+ |                                      
                    |  +------+ |  +------+                                      
                 +--|-----------+-----------+                                    
                 |  |           |           |                                    
                 |  |  +---+    |           |  +--------------+                  
                 |  |  |{A}|----+           |  |              |                  
                 |  |  +---+                |  | +---+        |                  
                 |  |    ^                  |  | |{A}|        |                  
                 |  |    |                  |  | +---+        |                  
                 |  |    | +--+     +-----+ |  |  ^           |                  
                 |  |    +-|{}|---->|{A,C}|--->|  |           |                  
                 |  |      +--+     +-----+ |  | +--+   +---+ |                  
                 | ++--+    |               |  | |{}|-->|{B}| |                  
                 | |{B}| <--+               |  | +--+   +---+ |                  
                 | +---+             "Real" |  |  "Simulated" |                  
                 |                      FSM |  |          FSM |                  
                 +--------------------------+  +--------------+                
   \end{verbatim}
    \caption{Example GECS FSM}
    \label{fig:gecs_fsm}
\end{figure}

In Figure \ref{fig:gecs_fsm}, the graph displays the following properties:
\begin{enumerate}
    \item There is always exactly one "real" context.
    \item All archetype vertices can contain edges to other archetype vertices.
    \item All simulated archetype vertices always contains an edge to the real context.
    \item All real archetype vertices always contain an edge to one simulated context.
    \item Each simulated context can only connect to one archetype from the real context.
    \item A simulated archetype cannot have an edge to another simulated archetype.
\end{enumerate}

The key feature of the GECS library is that each archetype simulates the events it wants to apply to the sequential world before the serialization point occurs. After the serialization point, the sequential world takes over and performs an analysis on the changes within each archtype to update itself before starting the next frame.

Effectively, the simulated worlds are the ledgers discussed in section \ref{sec:ledger} with the added benefit instead of caching operations to be done in parallel, the world simulates as if those changes are actually happening. So at the serialization point, the only operation that needs to be performed is a $\delta$ transition to another archetype for entities in the ledger, alognside new data.

\subsection{Data Structures}
\label{sec:ds}
The datastructures that compose the GECS library are thread-unsafe vectors and hashmaps. These datastructures are provided by another library self-developed called "csdsa", which contains C implementations of DSA fundamentals in C. 

The vector in csdsa wraps contiguous elements and provides a set of functional programming style utility functions to help with code clarity. The hashmap is implemented as an open address hashing table with a load factor of $0.75$. The set implementation is the same as the hashmap. There is nothing special about these structures. Originally, the hashmap was thread safe and was implemented using this paper \cite{hashmaps} but later found it unnecessary so it was scraped in development. 

\subsection{Allocator Design}
The allocator used internally by the GECS C Library is a customized stack allocator. This allocator was chosen because of how easy it is to implement concurrently. The stack allocator is not fancy in anyway, but supports creating stack frames in the heap to keep allocation requests minimized. GECS starts by allocating 8KB of memory for internal stack use. This memory is not used for allocating \texttt{g\_core\_t} members or other long-persisting objects. This memory is only used for internal manipulations and overall just available for routines when needed.

\subsection{World Struct}
The code defined in Figure \ref{code:g_core_t} is not exactly the same code as seen on the repository on GitHub. In the actual code, the types are different in order for the C compiler to do it's magic. The types presented here are for code clarity instead of compilability.

\texttt{g\_core\_t} represents a world from the concurrency model and it's what's given to the library user to manipulate the ECS.

\subsubsection{Flags}
The \texttt{is\_main} flag checks if this world is the original or if it's a ledger. This is used through various points in the implementation as the recursive breakpoint when doing manipulations to \texttt{g\_core\_t} objects. 

The \texttt{invalidate} flag will make more sense once the archetype struct is introduced but this exists because the \texttt{map\_t} type can grow. Each archetype maintains a cache of addresses for quick access to entity data. When the map resizes, those addresses become invalid and must be recache'd. Even though this is not ideal, the map grows by a factor of 2 each time the load factor is reached, so cache invalidations do not happen often. 

The \texttt{invalidate\_fsm} flag is a flag that gets set whenever a new archetype is added to the graph. This is because the systems must get redistributed across the graph.

\subsubsection{Data}
\texttt{component\_registery} and \texttt{archetype\_registry} are maps that utilize the hashing function. This is because component ids are generated via hashing and archetypes are sets of component ids. In order to ensure the same hash is generated for sets of components in differering orders, the hashes are first sorted before the archetype is passed into a hashing function. Unfortunately there was not enough time to implement a hashset and so ordered vectors are used in their place.

\begin{figure}[htbp]
    \begin{lstlisting}[
        language=C
    ]
struct g_core_t {
  int64_t tick;            /* The current tick being processed */
  atomic_uint_least64_t id_gen; /* Generates unique IDs */
    
  /* FLAGS */
  int8_t is_main;           /* Is "real" or "simulated" */
  int8_t invalidate_fsm;    /* Reprocess FSM before next tick */

  /* DATA */
  map_t component_registry; /* Map: hash(name) -> comp size */
  map_t archetype_registry; /* Map: hash([name]) -> archetype */
  map_t entity_registry;    /* Map: entt id -> archetype id AKA hash([name]) */
    
  vec_t system_registry;    /* Vec: system_data. */
};
    \end{lstlisting}
    \caption{\texttt{g\_core\_t} struct definition}
    \label{code:g_core_t}
\end{figure}

\subsection{Archetype Struct}
The code in Figure \ref{code:archetype} represents the archetype as seen in GECS. There's many members of this so the discussion of them will be split into two groups: The implementation members and the ledger/caching members. 

\begin{figure}[H]
    \begin{lstlisting}[
        language=C
    ]
struct archetype {
  gid archetype_id; /* Unique Identifier for this archetype. */
  
  set_t type_set;      /* Set : [hash(name)] */
  vec_t composite;     /* Contiguous vector of interleaved compents */
  vec_t contenders;    /* Vec: system_data */
  
  /* Entity lookup */
  map_t entt_positions; /* entt id -> index in composite */
  map_t offsets; /* Map: hash(name) -> interleaved comp offset */
  
  /* For concurrency and caching purposes. */
  g_core_t *simulation;          /* OOB mutations go here. */
  
  vec_t     entt_creation_buffer;  /* Queue : entt id */
  vec_t     entt_deletion_buffer;  /* Queue : entt id */
  vec_t     entt_mutation_buffer;  /* Queue : entt */
  vec_t     dead_fragments;          /* Vec : index_of(composite) */
};
    \end{lstlisting}
    \caption{\texttt{archetype} struct definition}
    \label{code:archetype}
\end{figure}

\subsubsection{Archetype Implementation Members}
The \texttt{type\_set} member is a set containing the hashes of type ID's for the components this archetype belongs to. Hashing the bytes of this member produces the \texttt{archetype\_id}. The \texttt{composite} vector is the vector where contiguous elements of this archetype are stored. Each element is a segment of memory in which components are interleaved. The \texttt{offsets} table is a table containing the offset of a particular component being queried on.  

The only member in this set related to concurrency is the \texttt{contenders} vector. This vector contains the list of systems that are processed by this archetype. Each system is processed sequentially. 

\subsubsection{Component Retrieval}
Since all archetypes live inside the composite, each archetype is provided a look-up table where the given component hash name is used to load how many bytes off from the start of the element the component exists on. In Figure \ref{code:component_retrieval}, Component A exists 2 bytes off from the composite element while Component C starts 5 bytes off from the composite element.

\begin{figure}[H]
\begin{verbatim}
Offset Table:                                                             
+-----------+-------+                                                     
|Component  |Offset |                                                     
+-----------+-------+                                                     
|Component A|     2 +------+                                              
+-----------+-------+      |                                              
|Component B|     0 +---+  |                                              
+-----------+-------+   |  |                                              
|Component C|     5 +---+--+--+                                           
+-----------+-------+   |  |  |                                           
                        |  |  |                                           
                        |  |  |                                           
Composite Vector:       v  v  v                                           
+--+--+-----+--+--+-----+--+--+-----+--+--+-----+--+--+-----+--+--+-----+ 
|  |  |     |  |  |     |  |  |     |  |  |     |  |  |     |  |  |     | 
|  |  |     |  |  |     |  |  |     |  |  |     |  |  |     |  |  |     | 
|B.|A.|C....|B.|A.|C....|B.|A.|C....|B.|A.|C....|B.|A.|C....|B.|A.|C....| 
+--+--+-----+--+--+-----+--+--+-----+--+--+-----+--+--+-----+--+--+-----+ 
|Index 0    |Index 1    |Index 2    |Index 3    |Index 4    |Index 5    | 
\end{verbatim}
\caption{Specific Component Retrieval}
\label{code:component_retrieval}
\end{figure}

\subsubsection{Archetype Caching And Ledger Members}
The rest of the members of the \texttt{archetype} struct are members related to caching and concurrency. For doing concurrent entity deletions and creations, the \texttt{entt\_deletion\_buffer} and \texttt{entt\_creation\_buffer} are used respectively. 

Entity transitions like adding and deleting components use the \texttt{entt\_mutation\_buffer}. When an entity gains a new component, the entity is marked to have gained said component by having its entity ID pushed into the mutation buffer. The component is then simulated in the \texttt{simulation}. When it is time to consolidate, the buffer is emptied one by one and their states are transitioned out of the simulation. 

The \texttt{dead\_fragments} vector is also used at consolidation time. This vector contains indices inside the composite vector of entities that no longer use this archetype. They were either deleted or transitioned out. The vector is sorted before use in the defragmentation algorithm.

\begin{figure}[H]
\begin{verbatim}
        +---------+     +----------------------------+                             
        |Real     |     | Archetype:                 |                             
        |Context  |     |                +---------+ |                             
        |         |     |                |Simulated| |                             
        |         |     | Entity         |Context  | |                             
        |         |     | Mutation       |         | |                             
        |     ----+--+  | Buffer:        |         | |                             
        |         |  |  | +----------+   |         | |                             
        |         |  |  | |Entity ID |   |         | |                             
        |         |  |  | +----------+   |         | |                             
        +---------+  |  | |5         |   |         | |                             
                     |  | |          | +-+--->     | |                             
                     |  | |1         | | |         | |                             
                     +--+-+--->  ----+-+ |         | |                             
                        | |13        |   |         | |                             
                        | +----------+   +---------+ |                             
                        |                            |                             
                        +----------------------------+                                                      
\end{verbatim}
\caption{Entity Mutation Interface}
\label{code:component_retrieval}
\end{figure}

\subsection{Systems}
The following is the \texttt{system\_data} struct. The \texttt{system\_data} struct contains a function pointer \texttt{run\_system} which is provided by the user. This is the entry point into the users code. The \texttt{requirements} set is a set of hashes component names where an archetype be a superset to be paired. See section \ref{sec:scheduling}.

\begin{figure}[H]
    \begin{lstlisting}[
        language=C
    ]
struct system_data {
  g_system  run_system;     /* A function pointer to the system. */
  set_t     requirements;   /* Vec: hash(name) */
};
    \end{lstlisting}
    \caption{\texttt{system\_data} struct definition}
    \label{code:sd_and_er}
\end{figure}

\subsection{API Objects}
There are three API structs that library users interface with: The \texttt{g\_pool}, \texttt{g\_par}, and the \texttt{g\_query} structs. 

\texttt{g\_query} is passed as the single parameter to all system functions. Using this struct, a user contextually interfaces with the system currently in process. It contains two pointers: a pointer to the world context and a pointer to the archetype.

\texttt{g\_pool} is used by a user who wants to begin querying over components inside the current archetype. It's a context object on how far iteration has progressed. Internally, it's actually just a wrapper over \texttt{g\_pool} and adds an index counter.

\texttt{g\_par} is used by a user who confidently want to do embarassingly parallel tasks using the libaries concurrency functions. It contains references to the offset table, the composite, the archetype, and the current tick.


\begin{figure}[htbp]
    \begin{lstlisting}[
        language=C
    ]
struct g_par {
  vec_t        *stored_components; /* Vector : fragment */
  map_t        *component_offsets; /* Map : hash(comp id) -> pos */
  archetype    *arch;
  int64_t       tick;
};

struct g_pool {
  gint64 idx;
  g_par  entities;
};

struct g_query {
  g_core    *world_ctx;
  archetype *archetype_ctx;
};
    \end{lstlisting}
    \caption{API Object definitions}
    \label{code:apis}
\end{figure}

\subsection{API}
The GECS API is roughly inspired by the FLECS API in how it accesses specific components. Otherwise it is of this papers own design. There are four categories of functions in the GECS API:
\begin{itemize}
    \item World Functions
    \item Thread Unsafe ECS Functions
    \item Thread Safe ECS Functions
    \item Query Functions
\end{itemize}

The header file containing the ECS defintions is linked in the Appendix \ref{appendix:a}.

\subsection{World Functions}
There are three world functions: \texttt{g\_create\_world}, \texttt{g\_progress}, \texttt{g\_destroy\_world} and each are self explanatory. \texttt{g\_progress} runs the ECS engine for exactly 1 tick.

\subsection{Thread Unsafe Functions}
The thread unsafe functions include any function that directly modifies the sequential worlds data structures. Either the maps, vectors, or \texttt{id\_gen}. There exists a thread unsafe function for all the normal entity operations: add, get, set, and remove components. This also includes the create and delete entity operations as well.  

\subsection{Thread Safe Functions}
Just like how there exists a thread unsafe function for all normal entity operations, there exists a thread safe function for all normal entity operations as well. It's also important to note that all query functions are also thread safe.

\subsubsection{Query Functions}
The query functions are the set of functions used to interface and progress the runtime. The two core functions are: \texttt{gq\_vectorize} and \texttt{gq\_seq}. Depending on how the user wants their components processed, they can use the vectorize function to enable concurrent element processing on the contiguous vector or they can use the sequential function to retrieve an iterator. There is nothing stopping the user at this point on creating their own concurrent model inside the system. \texttt{gq\_vectorize} is there only for utility.

\textbf{\texttt{gq\_vectorize} approach:} When using the vectorize function, the only function that can use this structure is the \texttt{gq\_each} function. This function splits the composite to be processed over multiple threads for the user automatically.

\textbf{\texttt{gq\_seq} approach:} This function returns \texttt{g\_pool} structure, allowing the user to process components however they please. To progress to the next element the user invokes \texttt{gq\_next} and to check if there are any elements left the user invokes \texttt{gq\_done}.if there are any elements left call \texttt{gq\_done}.

To select a component, pass in the \texttt{g\_pool} struct with the component name to \texttt{gq\_field}. Alternatively, pass into \texttt{gq\_is\_id} to check against important IDs and to load the hidden archetype components for this entity. 

\subsection{Examples}
In Appendix \ref{appendix:code_example_1}, there exists a demo game showing the usage of GECS. It's a re-creation of that simple chromium browser game seen when there's no internet connection. The majority of the systems run in parallel. The following is a "screenshot" of the game in action:

\begin{figure}[H]
    \begin{verbatim}
===============
│             │
│             │
│ Y         0 │
===============
Health: 19
    \end{verbatim}
\caption{Entity Cache Interface}
\label{code:component_retrieval}
\end{figure}

The game uses the \texttt{ncurses} library for rendering to the terminal. 

\subsection{Entity ID Representation}
In GECS, only entities use ID generation only one ID generator is used throughout the runtime. The ID is syncrhonized using the C atomics library. The reasoning behind \texttt{id\_gen} being least 64 and not fast 64 is because of the 64 byte guarantee was originally important to the library as it did smart ID generation in earlier versions. The last bit used to be reserved to know which context an entity existed in, either real or simulated. But this caused an ID vanishing issue because the user would not get the correct ID returned to them when creating entities concurrently. This essentially meant that any entities made concurrently will not be accessable to the library user.

To mitigate this issue, the ID generator was made atomic. It is kept as least for future development. In the future, GECS should have related entity pools placed in the first 16 bits, much like how FLECS does it.

\texttt{gid.c} contains the implemenation of how GECS increments atomics.

\subsection{The Tick Lifecycle}
The tick lifecycle is processing exactly one tick, which means executing the \texttt{g\_progress} function completely. The cycle is split into three parts: preparation, execution, and cleanup. Both the preparation and cleanup phases are done sequentially on the main thread, while the execution phase is where threads are scheduled to run systems.

The preparation phase only has one job which is to ensure the validity of the entity FSM before systems start transitioning entities over it. This is done by checking the \texttt{reprocess\_fsm} flag.

The execution phase starts by collecting all archetypes and iterating over them to reach their \texttt{contenders} vector. This vector contains pointers to the \texttt{system\_data} structs. Once loaded, threads will be scheduled to run the system process.

The cleanup phase is where the majority of the work occurs for GECS. It must do 4 things:
\begin{itemize}
    \item Check if fsm is invalidated and set invalidate flag
    \item Migrate entities out of their simulations
    \item Process the buffers and clear them to prepare for the next tick
    \item Defragment all composites that have unreference-able indices.
\end{itemize}

The cleanup phase algorithms for defragmentation and cache clearance are written into this section. The following sections will contain the algorithms for entity migration, Entity FSM processing, and scheduling that were referenced here.

\subsubsection{Defragmentation}
Defragmentation is the process of re-vectorizing a composite such that all elements in the vector are referenced by one lookup. In order to achieve this: the entities who have either been deleted or transitioned away from this archetype must have their composite index loaded and then removed from the vector. There are two parts to the defragmentation algorithm.

Because classical vector deletion is $O(N)$ time, if $e$ amount of entities are deleted, then the defragmentation algorithm will take $O(N^e)$ which is not feasible. As such, the vector \texttt{dead\_fragments} is used. When an entity leaves this archetype, the dead index is pushed into \texttt{dead\_fragments}. Then, at cleanup, the following algorithm is used to defragment the vector in $O(log(e) + N)$ time for $e$ indices.

\begin{figure}[H]
    \begin{enumerate}
        \item Sort \texttt{dead\_index}
        \item Create vector \texttt{out}
        \item Create vector \texttt{rolling\_offsets} and integer \texttt{dead\_index}.
        \item For each element $ e \in \texttt{composite}$, do the following 5-9:
        \item If the index of $e$ is dead, do the following steps 6-8:
        \item Increment \texttt{dead\_index}
        \item Push $\texttt{rolling\_offsets} \leftarrow \texttt{dead\_index}$
        \item Continue
        \item Push $\texttt{rolling\_offsets} \leftarrow \texttt{dead\_index}$
        \item Replace \texttt{composite} with \texttt{out}
    \end{enumerate}
    \caption{Defragment Algorithm Part 1}
    \label{alg:defrag1}
\end{figure}

In other words, Figure \ref{alg:defrag1}, produces a vector called the \texttt{rolling\_offset} vector that counts up by one each time a dead fragment index is reached. This is then used by part two of the algorithm to subtract all entity positions on this archetype by its associated value at the same index in \texttt{rolling\_offset}.
                  
\begin{figure}[H]
\begin{verbatim}
    Original                                        Dead                     
    Vector                                          Fragments                
    +-----+-----+-----+-----+-----+-----+-----+     +-----+                  
    |  A  |  B  |  C  |  D  |  E  |  F  |  G  | +---+  2  |                  
    +-----+-----+-----+-----+-----+-----+-----+ |   +-----+                  
    0     1     2     3     4     5     6       | +-+  3  |                  
                +-------------------------------+ | +-----+                  
                |                                 | |  5  |                  
    Offset      |     +---------------------------+ +--+--+                  
    Vector      v     v                                |                     
    +-----+-----+-----+-----+-----+-----+-----+        |                     
    |  0  |  0  |  2  |  2  |  2  |  3  |  3  |        |                     
    +-----+-----+-----+-----+-----+-----+-----+        |                     
    0     1     2     3     4     5     6              |                     
                                  ^                    |                     
                                  +--------------------+   
\end{verbatim}
\caption{Offset Vector Diagram}
\label{code:component_retrieval}
\end{figure}

The second part of the algorithm loads the positions and then performs the following: 

\begin{figure}[H]
\begin{verbatim}                                                                 
 Offset Vector                                                                 
   0+-----+     Entity                             New Entity             
    |  0  |     Positions                          Positions                        
   1+-----+     +-----+-----+                      +-----+-----+            
    |  0  |     |ID   |Index|                      |ID   |Index|            
   2+-----+     +-----+-----+ index - offset(index)+-----+-----+            
    |  2  |     |0    |1    +--------------------->|0    |  1  |            
   3+-----+     +-----+-----+ index - offset(index)+-----+-----+            
    |  2  |     |1    |4    +--------------------->|1    |  2  |            
   4+-----+     +-----+-----+ index - offset(index)+-----+-----+            
    |  2  |     |2    |6    +--------------------->|2    |  3  |            
   5+-----+     +-----+-----+                      +-----+-----+            
    |  3  |     |. . .|. . .|                      |. . .|. . .|            
    +-----+     +-----+-----+                      +-----+-----+            
 Original             |                     Defragmented |                  
 Vector+--------------+--------------+      Vector +-----+-----+            
       v                 v           v             v     v     v            
 +-----+-----+-----+-----+-----+-----+-----+ +-----+-----+-----+-----+      
 |  A  |  B  |  C  |  D  |  E  |  F  |  G  | |  A  |  B  |  E  |  G  |      
 +-----+-----+-----+-----+-----+-----+-----+ +-----+-----+-----+-----+      
 0     1     2     3     4     5     6       0     1     2     3            
\end{verbatim}
\caption{Update Entity Record With Offset Mapping Diagram}
\label{code:component_retrieval}
\end{figure}

This was part two of the defragmentation algorithm. At the end of this procedure, the composite vector is completely vectorized and ready for the next tick.

\subsubsection{Simulation Clearance Algorithm}
Simulation clearance is fairly straightforward since all that needs to be done is apply the \texttt{clear} function to all vectors and maps existing in simulated contexts. This is done because there should be no interference between ticks. All data before the end of a tick is transitioned out of the simulation anyway, so it is safe to clear without the risk of losing data.

For archetypes:
\begin{itemize}
    \item entt\_delete\_queue
    \item entt\_creation\_queue
    \item entt\_mutation\_queue
\end{itemize}

For simulated worlds:
\begin{itemize}
    \item entity\_registry
    \item archetype\_registry
    \item Everything in the simulated archetypes
\end{itemize}

\subsection{Entity FSM}
The entity FSM is a collection of internal functions in the \texttt{archetypes.c, gecs.c} files. The most important of which is the \texttt{delta\_transition} function, which given a world, an id, and a set of types, performs an entity state transition.

\subsubsection{Entity $\delta$ Transitions}
Archetype transitions roughly follow the algorithm defined in chapter \ref{chap:2}. The algorithim goes as follows:

\begin{figure}[htbp]
    \begin{enumerate}
        \item Given world $w$, entity ID $e \in w.\texttt{entity\_registry}$, and a set of types $t \in \mathcal{P}(w.type\_set)$
        \item Create the new archetype if this is the first time seeing set $t$
        \item Load $a \leftarrow$ next archetype
        \item If entity $e$ is $\emptyset$, perform Algorithm \ref{alg:empty_to_a} and exit
        \item Push an empty composite element into $a$
        \item Copy over specifically all types that were retained from the old archetype into the new archetype (perform $a.\texttt{composite} \leftarrow a.\texttt{type\_set} \cap old_a.\texttt{type\_set}$)
        \item Update entity record $e$ to have the last index in the \texttt{composite} and archetype pointer to $a$
        \item Update the old archetype's \texttt{dead\_fragments} member to contain $e$ since $e$ left the old archetype.
    \end{enumerate} 
    \caption{Entity General Transitions Algorithm}
    \label{alg:transition}
\end{figure}

\begin{figure}[H]
\begin{enumerate}
    \item Given archetype $a$, and entity $e$ who is in archetype $\emptyset$ 
    \item Push an empty fragment into composite of $a$,
    \item Update entity positions of a to include $e$ with last index in the \texttt{composite}
\end{enumerate}
\caption{Entity $\emptyset \rightarrow a$ Transition Algorithm}
\label{alg:empty_to_a}
\end{figure}

It's important to note that algorithms \ref{alg:transition} and \ref{alg:empty_to_a} are not thread safe but are still used in concurrent contexts. This is possible because simulations and the real context both contain the same members. So this algorithm is also used by the ledger to simulate the following entities positions.

\subsubsection{Concurrent Entity Component Additions and Access}
To add a new component while a system is running is actually remarkably simple. Since adding a new component guarantee's a $\delta$-transition, all we do is do a $\delta$ transition inside the simulated context. This is equivalent to the library user adding a component, the same function is used but its applied to the archetypes simulated world instead. 

Accessing is the same story as adding a new component. In the situation there exists the same component in the simulation and in the real context's currently processing archetype, GECS will always prefer the component in the real context. This is in order to preserve vectorizability and minimize moves out of the simulation.

\subsubsection{Concurrent Entity Component Deletions}
The deletion algorithm is a tricky one due to the use case of deleting components that existed when tick started. This means when deleting, a genuine transition is required because if this entity stayed at the same archetype, it risks giving access to undefined memory. To avoid this, when deleting original components off the entity, the entire entity is offloaded into the simulatiton and thats where the state transition occurs. 

Clearly, this makes a huge mess of things but it's nothing that cannot be recovered from. GECS uses a preference system when applying component access. The components on the archetype will stay fresh while the ones that were used to simulate the deletion transition inside the simulation will become stale. 

\begin{figure}[htbp]
    \begin{enumerate}
        \item Given a runtime query struct $q$, entity ID $e$, and set of types $t \in \mathcal{P}(q.\texttt{component\_registry})$
        \item Check if the simulation archetypes types of $e$ intersect with the entities current types. If they do not, this is the first time doing a deletion transition. Do the following steps for the first time, else skip to step 5:
        \item Construct a type union between the simulation and archetypes type set. 
        \item Transition with the type union into the simulation
        \item Apply the thread unsafe component removal function to the simulation and exit
\end{enumerate}
\caption{Entity Concurrent Component Removal}
\label{alg:conc_rem}
\end{figure}

\subsubsection{Entity Consolidation Algorithm}

The entity consolidation algorithm is used to clear the cache of mutations saved over the progress of a tick. These mutations exist in the buffers: entity creation, deletion, and mutation. This set of algorithms clears these three buffers and migrates all changes into the real context. Since GECS prefers to always use component data from the original archetype composite, there is a division between fresh components and stale components. The deletion operation is the only operation that produces these stale components. With this in mind, GECS performs the following algorithm to merge entities:

\begin{figure}[H]
    \begin{enumerate}
        \item Given the sequential world $w$
        \item For each archetype $a \in w.\texttt{archetype\_registry}$, perform the following steps:
        \item Perform Algorithm \ref{alg:entt_del_queue}
        \item Perform Algorithm \ref{alg:entt_cre_queue}
        \item Perform Algorithm \ref{alg:entt_merge}
\end{enumerate}
\caption{World Merge Algorithm}
\label{alg:world_merge}
\end{figure}

\begin{figure}[htbp]
    \begin{enumerate}
        \item While the delete queue is not empty:
        \item $e \leftarrow \texttt{pop()}$
        \item Remove $e$ from the the worlds member: \texttt{entity\_registry} 
        \item Remove $e$ from $e$'s archetype member: \texttt{entt\_positions} 
\end{enumerate}
\label{alg:entt_del_queue}
\caption{Consuming Entity Deletion Queue}
\end{figure}

\begin{figure}[htbp]
    \begin{enumerate}
        \item While the creation queue is not empty:
        \item $e \leftarrow \texttt{pop()}$
        \item If entity already exists in the worlds $\texttt{entity\_registry}$, continue
        \item Transition $e$ to be on the same archetype in the real context as it is in the $e$ simulated context
        \item Set all the components to have the same data as well
\end{enumerate}
\caption{Consuming The Entity Creation Queue}
\label{alg:entt_cre_queue}
\end{figure}

\begin{figure}[htbp]
\begin{enumerate}
    \item While the mutation queue is not empty:
    \item $e \leftarrow \texttt{pop()}$
    \item Load the simulated archetype of $e$ into $a\_{sim}$ and the real archetype of $e$ into $a\_{real}$
    \item Perform the type union between both archetypes into $x \leftarrow a_{sim} \cap a_{real}$
    \item Transition $e$ to archetype $x$
    \item Move over only new component data to $x$, ignoring types that already exist on $a_{real}$
\end{enumerate}
\caption{Entity Merge}
\label{alg:entt_merge}
\end{figure}

\subsection{Scheduling}
Scheduling is done the exact way as stated in chapter \ref{chap:2} and is no different in the implementation. Thus, this section is omitted.

\subsection{Support For Embarassingly Parallel Tasks}
GECS supports embarassingly parallel tasks my dividing dividing an archetypes composite into chunks to run on separate threads. All this is handled by GECS and the user only needs to pass in a function they deem thread safe.
% \section{Benchmarks}
Due to the scope of this project there was not much time left for benchmarks and generating benchmark statistics. Most benchmarking efforts found online were not benching concurrent behaviors that this engine would excel at. For example, the most comprehensive set of benchmarking statistics can be found in reference \cite{ECS_benchmark}. The system benchmarks do not test the concurrency of systems and place emphasis on the performance of the libraries batch processing implementation. GECS supports very minimal batch processing so the library cannot compete. It's also worth noting that after a certain threshold of about 256 entities, other libraries start processing entity requests through a different way.

It's also important to note I could not get the code from \cite{ECS_benchmark} working on my local machine. The GECS values are from local benchmark efforts while the rest are on their directly from another dataset.

The following are all benchmarks:

\begin{table}[htbp]
    \centering
    \begin{tabular}{lllll}
        Entity Count & GECS      & EnTT   & FLECS    &  \\
        1            & 98000ns   & 2868ns & 401342ns &  \\
        4            & 117000ns  & 3812ns & 405156ns &  \\
        8            & 413000ns  & 3467ns & 407955ns &  \\
        16           & 855000ns  & 3801ns & 408108ns &  \\
        32           & 1875000ns & 4476ns & 414509ns &  \\
        64           & 13485us   & 5742ns & 422601ns &  \\
        256          & 155664us  & 13us   & 486us    &  \\
        $\sim$1k     & 2030297us & 100us  & 740us    & 
    \end{tabular}
    \caption{Create N Entities With 2 Components}    
\end{table}

\begin{table}[H]
    \centering
    \begin{tabular}{lllll}
        Entity Count & GECS      & EnTT   & FLECS    &  \\
        1            & 65000ns   & 1053ns & 349451ns &  \\
        4            & 116000ns  & 1123ns & 350054ns &  \\
        8            & 205000ns  & 1312ns & 351086ns &  \\
        16           & 382000ns  & 1609ns & 351077ns &  \\
        32           & 693000ns & 2425ns & 351619ns &  \\
        64           & 1378000us   & 3701ns & 353878ns &  \\
        256          & 6267000us  & 11us   & 362us    &  \\
        $\sim$1k     & 2231600us & 43us  & 400us    & 
    \end{tabular}
    \caption{Destroy N Entities With 2 Components}    
\end{table}

The destroy table also includes in it the entity creation time. So the time may be slightly less over each one.

\begin{table}[H]
    \centering
    \begin{tabular}{lllll}
    Entity Count & GECS      & EnTT   & FLECS    &  \\
    1            & 65us   & 3ns & 26ns &  \\
    4            & 153us  & 12ns & 105ns &  \\
    8            & 381us  & 25ns & 205ns &  \\
    16           & 720us  & 50ns & 400ns &  \\
    32           & 1142us & 104ns & 800ns &  \\
    64           & 5989us   & 204ns & 1596ns &  \\
    256          & 23489us  & 0us   & 6us   &  \\
    $\sim$1k     & 406ms & 3us  & 25us  & 
    \end{tabular}
    \caption{Read N Entities With 2 Components}    
\end{table}

% \section{Discussion}
I am still working on benchmarks against other ECS's. Once that's ready I will write this section.
% \section{Conclusion}
To conclude, the question of "Can you make a wait-free ECS?" was answered as this paper proposes a model that is wait-free. What started off as a practical project, which was intended to be the main meat of the paper, led to the development of some interesting theory. The difficulties in benchmarking the model show that this ECS is a little unique in its concurrency architecture and that it is missing key production features. In my opinion, I do not believe the benchmark results are accurate measures of this engine's strengths because the benchmarks focused on processing as many entities as quickly as possible in specialized settings. This engine, with its finite state machine, has the ability to grow, which is unique. A more accurate test for GECS would be to implement the same set of fairly complex game systems in different engines and test via throughput on how fast a tick processes.

In summary, while the proposed wait-free ECS model shows promise and introduces an alternative approach, further work is necessary to refine its entity throughput and to establish more comprehensive benchmarking methods. More ongoing research could lead to improvements to the efficiency of simulations, potentially influencing future developments.
\newpage

\printbibliography

\end{document}