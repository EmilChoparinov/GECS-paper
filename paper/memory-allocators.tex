\section{Memory Allocators}
\label{sec:memory-allocators}

Because ECS systems are expected to withstand thousands of entities being added
and deleted. We need an allocator that is capable of doing minimal memory 
requests and distribute memory within userspace.

\subsection{Arena Allocator}
\label{subsec:arena-allocator}

An arena allocator is an allocator that ensures all allocations are cheap 
and allows for the utility of connecting object lifetimes to other allocated 
objects. This means we no longer need to keep track of each items allocations
individually and only need to free the arena when the time comes.

\subsubsection{How does the arena handle mallocs?}
Whenever a malloc occurs, the arena has a pointer to large contiguous piecees of
memory and returns the next pointer with the amount required and increments the
availability pointer.

\subsubsection{What if we run out of memory?}
If the arena fills up, it will request a new block of memory with the same size.
The locations of these blocks are tracked with a hashmap in memory. So when the
time comes to free, all are freed together. 

\subsubsection{What if we attempt to allocate something larger than the block size?}
In such a case the allocator will personally allocate a new region of that exact size.

\subsection{How do we handle concurrency?}
The way it works for now is with costly mutex locks. On any allocation event, 
a thread must aquire the lock, do their job, and release.

\subsection{Arena As Stack Allocator}
\label{subsec:stack-allocator}

A stack allocator is similar to an arena except the lifetime of the allocator is a the end of the stack frame. We can 
simulate this easily by building on top of the arena allocator. Note: This has not been made yet because it may not be
needed.